\documentclass{article}

% Symbols
\usepackage{amsfonts, amsthm}
\usepackage{upgreek}
\usepackage{physics}
\usepackage{cancel}
\usepackage{amssymb, latexsym, amsmath}

%Algorithms
\usepackage[ruled,lined,linesnumbered,commentsnumbered]{algorithm2e}

%% Identación
\setlength{\parindent}{0cm}

% Código
\newcommand{\code}[1]{\textcolor{white!25!black}{\texttt{#1}}}
\usepackage{listings}

%AMS
\usepackage{amsthm}
\newtheorem{algo-thm}{Algoritmo}

% Proof
\renewcommand*{\proofname}{\textbf{Demostraci\'on:}}
% Theorem
\newtheorem*{theorem}{Teorema}

% Graphics
\usepackage{graphicx}
\usepackage{pgf}

% Color a letras.
%\usepackage[usenames,dvipsnames,svgnames,table]{xcolor}

% Tikz
\usepackage{tkz-graph}
\usepackage{tikz}
\usetikzlibrary{arrows,automata}
\usepackage{tikz}
\usetikzlibrary{arrows,automata}
%\usetikzlibrary[topaths]

% Def. Dr. César.
\usetikzlibrary{shapes,calc}
\tikzstyle{edge}=[shorten <=2pt, shorten >=2pt, >=stealth, line width=1.1pt]
\tikzstyle{blueE}=[shorten <=2pt, shorten >=2pt, >=stealth, line width=1.5pt, blue]
\tikzstyle{blackV}=[circle, fill=black, minimum size=6pt, inner sep=0pt, outer sep=0pt]
\tikzstyle{blueV}=[circle, fill=blue, draw, minimum size=6pt, line width=0.75pt, inner sep=0pt, outer sep=0pt]
\tikzstyle{redV}=[circle, fill=red, draw, minimum size=6pt, line width=0.75pt, inner sep=0pt, outer sep=0pt]
\tikzstyle{redSV}=[semicircle, fill=red, minimum size=3pt, inner sep=0pt, outer sep=0pt, rotate=225]
\tikzstyle{blueSV}=[semicircle, fill=blue, minimum size=3pt, inner sep=0pt, outer sep=0pt, rotate=225]
\tikzstyle{blackSV}=[semicircle, fill=black, minimum size=3pt, inner sep=0pt, outer sep=0pt, rotate=225]
\tikzstyle{vertex}=[circle, draw, minimum size=6pt, line width=0.75pt, inner sep=0pt, outer sep=0pt]

% Margins
\addtolength{\voffset}{-1.5cm}
\addtolength{\hoffset}{-1.5cm}
\addtolength{\textwidth}{3cm}
\addtolength{\textheight}{3cm}

%Header-Footer
\usepackage{fancyhdr}
\renewcommand{\headrulewidth}{1pt}

\newcommand{\set}[1]{
  \left\{ #1 \right\}
}

%\pagenumbering{gobble} -- Este comando
%                       -- quita el número de página.
\footskip = 50pt
\renewcommand{\headrulewidth}{1pt}

\pagestyle{fancyplain}

\begin{document}
\title{UNIVERSIDAD AUT\'ONOMA DE M\'EXICO\\ Facultad de Ciencias}
\author{Autores:
  \\ Fernanda Villaf\'an Flores
  \\ Fernando Alvarado Palacios
  \\ Adri\'an Aguilera Moreno}
\date{}
\maketitle
\begin{center}
  \includegraphics[scale=0.20]{../Imagen/Portada.jpg}\\[0.4cm]
  \Large
  \bf{Gr\'aficas y Juegos}
  \normalsize
\end{center}
\newpage
\fancyhead[r]{ Gr\'aficas y Juegos 2022-1}
%%%%%%%%%%%%%%%%%%%%%%%%%%%%%%%%%%%%%%%%%%%%%%%%%%%%%
\section*{\LARGE{Tarea 13}}
\begin{enumerate}
  %% EJERCICIO 1
  \item Considere los v\'ertices $x = (0, \dots, 0)$, y $y = (1, \dots, 1)$
  $n$-cubo, $Q_n$.   Describa una colecci\'on m\'axima de $xy$-trayectorias
  ajenas por aristas dos a dos en $Q_n$ y un $xy$-corte m\'inimo por
  v\'ertices que separe a $x$ de $y$.

  %% EJERCICIO 2
  \item Demuestre que una gr\'afica $3$-conexa no bipartita contiene al menos
  cuatro ciclos impares.

  %% EJERCICIO 3
  \item[3.] Sea $G$ una gr\'afica $k$-conexa con $k \ge 2$.   Demuestre que
    cualesquiera $k$ v\'ertices de $G$ est\'an contenidos en un ciclo com\'un.

    \begin{proof}
      Sea $x \in V_{G}$ y sea $C$ un subconjunto de $G$ tal que $V_{C} = V_{G} - \{x\}$.

      Supongamos que $x \notin C$.

      Como $G$ es $k$-conexa, por el \textbf{Lema del Abanico} sabemos que existen $k$-trayectorias internamente ajenas entre $x$ y $C$.

      Sean $v_{1}, v_{2}, \dots, v_{k}$ los vértices de $C$ que alcanzó $x$ por medio de las $k$-trayectorias.

      Ahora, sabemos por la \textbf{Proposición 1.6.1} de las \textit{Notas de Clase} que:
      $$\text{\textit{Para una gráfica $G$, si $\delta \leq 2$ entonces $G$ tiene al menos un ciclo.}}$$

      Si hacemos a $C$ un ciclo de $G$, entonces $C$ se ve de la siguiente manera:
      $$C = (v_{1}, \dots, v_{2}, \dots, v_{k}, \dots, v_{1})$$
      Como $x$ es adyacente a $v_{1}, v_{2}, \dots, v_{k}$ vértice de $C$, tomemos el siguiente ciclo:
      $$C' = (x, v_{1}, v_{2}, \dots, v_{k}, x)$$
      Por lo tanto, probamos que cualesquiera $k$ vértices de $G$ están contenidos en un ciclo común.
    \end{proof}

  %% EJERCICIO 4
  \item Sea $G$ una gr\'afica $k$-conexa, y sean $X$ y $Y$ subconjuntos de $V$
  de cardinalidad al menos $k$.   Demuestre que existe una familia de al menos
  $k$ $(X,Y)$-trayectorias ajenas.

  %% EJERCICIO 5
  \item Considere el siguiente juego.   Dos jugadores seleccionan alternadamente
  v\'ertices distintos $v_0, v_1, v_2, \dots$ sobre una gr\'afica $G$, donde
  para cada $i \ge 0$, $v_{i+1}$ es adyacente a $v_i$.   El \'ultimo jugador
  que pueda elegir un v\'ertice es el ganador. Demuestre que el primer jugador
  tiene una estrategia ganadora si y s\'olo si $G$ no tiene un apareamiento
  perfecto.

  %% EJERCICIO 6
  \item Demuestre que es imposible, utilizando fichas de domin\'o, cubrir un
  tablero de ajedrez donde dos esquinas contrarias fueron removidas.

  %% EJERCICIO 7
  \item Una l\'inea en una matriz es un rengl\'on o una columna de la matriz.
  Demuestre que el m\'inimo n\'umero de l\'ineas que contienen todas entradas
  distintas de cero en una matriz es igual al m\'aximo n\'umero de entradas
  distintas de cero tales que no hay dos en la misma l\'inea.

  %% EJERCICIO 8
  \item[8.] Demuestre que para cada $k > 0$, toda gr\'afica bipartita y $k$ regular
  es $1$-factorizable.

    \begin{proof}
      Procedemos por inducción sobre $k$.

      \begin{itemize}
        \item \textbf{Caso Base:} $k = 1$.

          Sea $G$ una gráfica $1$-regular.

          Por la definición de $1$-factorizable, tenemos que $G$ es el factor que estábamos buscando.

        \item \textbf{Hipótesis de Inducción.} Supongamos que para cada $k > 0$ se cumple que toda gráfica $k$-regular es $1$-factorizable.

        \item \textbf{Paso Inductivo.}

          Sea $G$ una gráfica ($k + 1$)-regular.

          Dado que $G$ es bipartita, por el \textbf{Teorema 1.7.2} sabemos que $G$ no contiene ciclos impares. Esto implica que cada vértice en $G$ tiene grado par y por tanto, $G$ es par. \\
          Así, entonces $G$ es euleriana y por el \textbf{Teorema 4.1.1} sabemos que $G$ se puede descomponer en una familia de ciclos $C_{1}, C_{2}, \dots, C_{n}$.

          Notemos que si existe un ciclo $C_{i}$ que pase por todos los vértices, al eliminarlo el grado de todos los vértices disminuiría en $2$. Así, eliminamos a todos los ciclos necesarios para poder disminuir el grado de todos los vértices en $2$. Por lo que obtuvimos una gráfica $G'$ que es $k$-regular, la cual por \textbf{Hipótesis de Inducción} sabemos que es $1$-factorizable.

          Además, notemos que los factores $H_{1}, \dots, H_{n}$ de $G'$ también se van a encontrar en $G$; y la unión de los ciclos $C_{i}, \dots, C_{j}$ que borramos para obtener a $G'$ forman el factor $H_{n + 1}$ tal que $G = H_{1}, \dots, H_{n}, H_{n + 1}$.

          Por lo tanto, probamos que $G$ es $1$-factorizable.
      \end{itemize}
    \end{proof}
\end{enumerate}

\section*{\Large{Puntos Extra}}
\begin{enumerate}
  \item Sea $A_1, \dots, A_m$ un conjunto de subconjuntos de $S$. Un sistema
    de representantes distintos para la familia $(A_1, \dots, A_m)$ es un
    subconjunto $\{ a_1, \dots, a_m \}$ de $S$ tal que $a_i \in A_i$, $1 \le i
    \le m$, y $a_i \ne a_j$ para $i \ne j$. Demuestre que $(A_1, \dots, A_m)$
    tiene un sistema de representantes distintos si y s\'lo si
    \[
      \left| \bigcup_{i \in J} A_i \right| \ge |J|
    \]
    para cada subconjunto $J$ de $\{ 1, \dots, m \}$.

  \item Demuestre que son equivalentes
    \begin{enumerate}
      \item El Teorema del \textsc{Flujo-M\'aximo Corte-M\'inimo}.

      \item El Teorema de Menger.

      \item El Teorema de Hall.

      \item El Lemma de K\"onig.
    \end{enumerate}
\end{enumerate}
\end{document}
