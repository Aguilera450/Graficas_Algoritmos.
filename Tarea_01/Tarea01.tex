\documentclass{article}

%Symbols
\usepackage{recycle}
\usepackage{amsfonts}

%Margins
\addtolength{\voffset}{-1cm}
\addtolength{\hoffset}{-1.5cm}
\addtolength{\textwidth}{3cm}
\addtolength{\textheight}{2cm}

%Header-Footer
\usepackage{fancyhdr}
%Header Info
\lhead{Profesor: C\'esar Hern\'andez Cruz \\
       Ayudante: Mauricio Carrasco Ruiz}
\rhead{Gr\'aficas y Juegos 2022-1}
%Footer Info
\rfoot{\recycle}
\cfoot{\vspace{-0.8cm}?`Realmente necesitas imprimir esta hoja?}
\lfoot{\recycle}
\pagenumbering{gobble}
\footskip = 50pt
\renewcommand{\headrulewidth}{1pt}

\newcommand{\set}[1]{%
\left\{ #1 \right\}%
}

\pagestyle{fancyplain}

\begin{document}
\section*{\LARGE{Tarea 1}}

\begin{enumerate}

  \item Sea $n$ un entero, $n \ge 3$.   Demuestre que existe un \'unico
    $n$-ciclo, salvo isomorfismo.

  \item De un ejemplo de tres gr\'aficas del mismo orden, mismo tama\~no y misma
    sucesi\'on de grados tales que cualesquiera dos de dichas gr\'aficas no sean
    isomorfas, al menos una de ellas sea conexa, y al menos una sea inconexa.

  \item Sea $D$ una digr\'afica.   Demuestre que
    $$\sum_{v \in V_D} d^+(v) = \sum_{v \in V_D} d^-(v) = |A_D|.$$

  \item  Sea $n$ un entero positivo. Definimos a la {\em Ret\'icula Booleana},
    $BL_n$, como la gr\'afica cuyo conjunto de vértices es el conjunto de todos
    los posibles subconjuntos de $\set{1, \cdots, n}$, donde dos subconjuntos
    $X$ y $Y$ son adyacentes si y s\'olo si su diferencia sim\'etrica tiene
    exactamente un elemento.
    \begin{enumerate}
      \item Dibuje $BL_1, BL_2, BL_3$ y $BL_4$.

      \item Determine $|V_{BL_n}|$ y $|E_{BL_n}|$. (Justifique su respuesta.)

      \item Demuestre que $BL_n$ es bipartita para cualquier $n \in
        \mathbb{Z}^+$.
    \end{enumerate}


  \item Sea $G[X, Y]$ una gr\'afica bipartita.
    \begin{enumerate}
      \item Demuestre que $\sum_{v \in X} d(v) = \sum_{v \in Y} d(v)$.

      \item Demuestre que si $G$ es $k$-regular, con $k \ge 1$, entonces $|X| =
        |Y|$.
    \end{enumerate}

\end{enumerate}

\subsection*{Puntos Extra}

\begin{enumerate}
  \item Sea $G = [X, Y]$ una gr\'afica bipartita con $|X| = r$ y $|Y| = s$.
    \begin{enumerate}
      \item Demuestre que $|E| \le rs$.

      \item Deduzca que $|E| \le \frac{|V|^2}{4}$.

      \item Describa a las gr\'aficas bipartitas que cumplen la igualdad en el
        inciso anterior. Justifique su respuesta.
    \end{enumerate}

\end{enumerate}

\end{document}
