\documentclass{article}

% Symbols
%\usepackage{recycle}
\usepackage{amsfonts, amsthm}
\usepackage{upgreek}
\usepackage{physics}
\usepackage{cancel}
\usepackage{amssymb, latexsym, amsmath}

% Proof
\renewcommand*{\proofname}{\textbf{Demostraci\'on:}}

% Graphics
\usepackage{graphicx}
\usepackage{pgf}

% Color a letras.
%\usepackage[usenames,dvipsnames,svgnames,table]{xcolor}

% Tikz
\usepackage{tikz}
\usetikzlibrary{arrows,automata}
\usepackage{tikz}
\usetikzlibrary{arrows,automata}

\usetikzlibrary{shapes,calc}
\tikzstyle{edge}=[shorten <=2pt, shorten >=2pt,
  >=stealth, line width=1.1pt]
\tikzstyle{blueE}=[shorten <=2pt, shorten >=2pt,
  >=stealth, line width=1.5pt, blue]
\tikzstyle{blackV}=[circle, fill=black,
  minimum size=6pt,
  inner sep=0pt, outer sep=0pt]
\tikzstyle{blueV}=[circle, fill=blue, draw,
  minimum size=6pt, line width=0.75pt,
  inner sep=0pt, outer sep=0pt]
\tikzstyle{redV}=[circle, fill=red, draw,
  minimum size=6pt, line width=0.75pt,
  inner sep=0pt, outer sep=0pt]
\tikzstyle{redSV}=[semicircle, fill=red, minimum
  size=3pt, inner sep=0pt, outer sep=0pt,
  rotate=225]
\tikzstyle{blueSV}=[semicircle, fill=blue, minimum
  size=3pt, inner sep=0pt, outer sep=0pt,
  rotate=225]
\tikzstyle{blackSV}=[semicircle, fill=black, minimum
  size=3pt, inner sep=0pt, outer sep=0pt,
  rotate=225]
\tikzstyle{vertex}=[circle, draw, minimum size=6pt,
  line width=0.75pt, inner sep=0pt,
  outer sep=0pt]

% Margins
%Margins
\addtolength{\voffset}{-1cm}
\addtolength{\hoffset}{-1cm}
\addtolength{\textwidth}{2cm}
\addtolength{\textheight}{2cm}
%Header-Footer
\usepackage{fancyhdr}
\renewcommand{\headrulewidth}{1pt}

\newcommand{\set}[1]{%
  \left\{ #1 \right\}%
}

%\pagenumbering{gobble} -- Este comando
%                       -- quita el número de página.
\footskip = 50pt
\renewcommand{\headrulewidth}{1pt}

\pagestyle{fancyplain}

\begin{document}
\title{UNIVERSIDAD AUT\'ONOMA DE M\'EXICO\\ Facultad de Ciencias}
\author{Autores:
  \\ Fernanda Villaf\'an Flores
  \\ Fernando Alvarado Palacios
  \\ Adri\'an Aguilera Moreno}
\date{}
\maketitle
\begin{center}
  \includegraphics[scale=0.20]{../Imagen/Portada.jpg}\\[0.4cm]
  \Large
  \bf{Gr\'aficas y Juegos}
  \normalsize
\end{center}
\newpage
\fancyhead[r]{ Gr\'aficas y Juegos 2022-1}
\section*{\LARGE{Tarea 3}}


\begin{enumerate}
	\item Demuestre que si $e \in E$, entonces $c(G) \le c(G-e) \le c(G) + 1$.

	\item Una gr\'afica es \textit{escindible completa} si su conjunto de
    v\'ertices admite una partici\'on $(S,K)$ de tal forma que $S$ es un
    conjunto independiente, $K$ es un clan, y cada v\'ertice en $S$ es adyacente
    a cada v\'ertice en $K$.   Demuestre que una gr\'afica es escindible
    completa si y s\'olo si no contiene a $C_4$ ni a $\overline{P_3}$ como
    subgr\'afica inducida.

	\item \begin{enumerate}
			\item Demuestre que si $|E| > {|V|-1 \choose 2}$, entonces $G$ es conexa.

			\item Para $|V| > 1$ encuentre una gr\'afica inconexa con $|E| = {|V|-1
        \choose 2}$.
		\end{enumerate}

	\item \begin{enumerate}
			\item Demuestre que si $\delta > \lfloor \frac{|V|}{2} \rfloor - 1$,
        entonces $G$ es conexa.

			\item Para $|V|$ par encuentre una gr\'afica $(\lfloor \frac{|V|}{2}
        \rfloor -1)$-regular e inconexa.
		\end{enumerate}

	\item Demuestre que si $D$ no tiene lazos y $\delta^+ \ge 1$, entonces $D$
	 contiene un ciclo dirigido de longitud al menos $\delta^+ + 1$.
\end{enumerate}

\section*{Puntos Extra}
\begin{enumerate}
	\item Demuestre que el n\'umero de $v_i v_j$-caminos de longitud $k$ en $G$ es
    $(A^k)_{ij}$ donde $A$ es la matriz de adyacencia de $G$.

	\item Sea $G$ una gr\'afica bipartita de grado m\'aximo $k$.   Demuestre que
    existe una gr\'afica bipartita $k$-regular, $H$, que contiene a $G$ como
    subgr\'afica inducida.
\end{enumerate}


\end{document}
