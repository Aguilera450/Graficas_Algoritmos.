\documentclass{article}

% Symbols
%\usepackage{recycle}
\usepackage{amsfonts, amsthm}
\usepackage{upgreek}
\usepackage{physics}
\usepackage{cancel}
\usepackage{amssymb, latexsym, amsmath}

%Algorithms
\usepackage[ruled,lined,linesnumbered,commentsnumbered]{algorithm2e}

%% Identación
\setlength{\parindent}{0cm}

% Código
\newcommand{\code}[1]{\textcolor{white!25!black}{\texttt{#1}}}
\usepackage{listings}

%AMS
\usepackage{amsthm}
\newtheorem{algo-thm}{Algoritmo}

% Proof
\renewcommand*{\proofname}{\textbf{Demostraci\'on:}}

% Graphics
\usepackage{graphicx}
\usepackage{pgf}

% Color a letras.
%\usepackage[usenames,dvipsnames,svgnames,table]{xcolor}

% Tikz
\usepackage{tkz-graph}
\usepackage{tikz}
\usetikzlibrary{arrows,automata}
\usepackage{tikz}
\usetikzlibrary{arrows,automata}
%\usetikzlibrary[topaths]

% Def. Dr. César.
\usetikzlibrary{shapes,calc}
\tikzstyle{edge}=[shorten <=2pt, shorten >=2pt, >=stealth, line width=1.1pt]
\tikzstyle{blueE}=[shorten <=2pt, shorten >=2pt, >=stealth, line width=1.5pt, blue]
\tikzstyle{blackV}=[circle, fill=black, minimum size=6pt, inner sep=0pt, outer sep=0pt]
\tikzstyle{blueV}=[circle, fill=blue, draw, minimum size=6pt, line width=0.75pt, inner sep=0pt, outer sep=0pt]
\tikzstyle{redV}=[circle, fill=red, draw, minimum size=6pt, line width=0.75pt, inner sep=0pt, outer sep=0pt]
\tikzstyle{redSV}=[semicircle, fill=red, minimum size=3pt, inner sep=0pt, outer sep=0pt, rotate=225]
\tikzstyle{blueSV}=[semicircle, fill=blue, minimum size=3pt, inner sep=0pt, outer sep=0pt, rotate=225]
\tikzstyle{blackSV}=[semicircle, fill=black, minimum size=3pt, inner sep=0pt, outer sep=0pt, rotate=225]
\tikzstyle{vertex}=[circle, draw, minimum size=6pt, line width=0.75pt, inner sep=0pt, outer sep=0pt]

% Margins
\addtolength{\voffset}{-1.5cm}
\addtolength{\hoffset}{-1.5cm}
\addtolength{\textwidth}{3cm}
\addtolength{\textheight}{3cm}

%Header-Footer
\usepackage{fancyhdr}
\renewcommand{\headrulewidth}{1pt}

\newcommand{\set}[1]{
  \left\{ #1 \right\}
}

%\pagenumbering{gobble} -- Este comando
%                       -- quita el número de página.
\footskip = 50pt
\renewcommand{\headrulewidth}{1pt}

\pagestyle{fancyplain}

\begin{document}
\title{UNIVERSIDAD AUT\'ONOMA DE M\'EXICO\\ Facultad de Ciencias}
\author{Autores:
  \\ Fernanda Villaf\'an Flores
  \\ Fernando Alvarado Palacios
  \\ Adri\'an Aguilera Moreno}
\date{}
\maketitle
\begin{center}
  \includegraphics[scale=0.20]{../Imagen/Portada.jpg}\\[0.4cm]
  \Large
  \bf{Gr\'aficas y Juegos}
  \normalsize
\end{center}
\newpage
\fancyhead[r]{ Gr\'aficas y Juegos 2022-1}
%%%%%%%%%%%%%%%%%%%%%%%%%%%%%%%%%%%%%%%%%%%%%%%%%%%%%
\section*{\LARGE{Tarea 8}}
\begin{enumerate}
  %%%%%%%%%%%%%%%%%%%%%%%%%%%%%%%%%%%%%%%%%%%%%%%%%%%%%%%%%%%%%%%%%%%%%%%%% Ejercicio 01
\item Sean $G$ una gr\'afica conexa y $e \in E$.   Demuestre que
  \begin{enumerate}
    %%~~~~~~~~~~~~~~~~~~~~~~~~~~~~~~~~~~~~~~~~~~~~~~~~~~~ inciso (a)
  \item $e$ est\'a en cada \'arbol generador de $G$ si y s\'olo si $e$ es un puente
    de $G$;
        %%~~~~~~~~~~~~~~~~~~~~~~~~~~~~~~~~~~~~~~~~~~~~~~~~~~~ inciso (b)
  \item $e$ no est\'a en \'arbol generador alguno de $G$ si y s\'olo si $e$ es un lazo.
  \end{enumerate}
  
  %%%%%%%%%%%%%%%%%%%%%%%%%%%%%%%%%%%%%%%%%%%%%%%%%%%%%%%%%%%%%%%%%%%%%%%%% Ejercicio 02
\item Modifique el algoritmo BFS para que regrese una bipartici\'on de la
  gr\'afica (si la gr\'afica es bipartita) o un ciclo impar (si la gr\'afica no es bipartita).
  
  %%%%%%%%%%%%%%%%%%%%%%%%%%%%%%%%%%%%%%%%%%%%%%%%%%%%%%%%%%%%%%%%%%%%%%%%% Ejercicio 03
\item Describa un algoritmo basado en BFS para encontrar el ciclo impar m\'as
  corto en una gr\'afica.
  
  %%%%%%%%%%%%%%%%%%%%%%%%%%%%%%%%%%%%%%%%%%%%%%%%%%%%%%%%%%%%%%%%%%%%%%%%% Ejercicio 04
\item Sea $G$ una gr\'afica con conjunto de bloques $B$ y conjunto de
  v\'ertices de corte $C$.   La {\em gr\'afica de bloques y cortes} de $G$,
  denotada por $B_C (G)$, esta definida por $V_{B_C (G)} = B \cup C$ y
  si $u, v \in V_{B_C (G)}$, entonces $uv \in E_{B_C (G)}$ si y s\'olo si
  $u \in B$, $v \in C$ y $v$ es un v\'ertice de $u$.   Demuestre que
  $B_C (G)$ es un \'arbol.
  
  %%%%%%%%%%%%%%%%%%%%%%%%%%%%%%%%%%%%%%%%%%%%%%%%%%%%%%%%%%%%%%%%%%%%%%%%% Ejercicio 05
\item Describa un algoritmo para encontrar un bosque generador en una
  gr\'afica arbitraria (no necesariamente conexa).
  
  %%%%%%%%%%%%%%%%%%%%%%%%%%%%%%%%%%%%%%%%%%%%%%%%%%%%%%%%%%%%%%%%%%%%%%%%% Ejercicio 06
\item Una {\em gr\'afica de Moore de di\'ametro $d$} es una gr\'afica
  regular de di\'ametro $d$ y cuello $2d+1$.   Demuestre que si $G$ es
  una gr\'afica de Moore, entonces todos los \'arboles de BFS de $G$
  son isomorfos.

\end{enumerate}
%%%%%%%%%%%%%%%%%%%%%%%%%%%%%%%%%%%%%%%%%%%%%%%%%%%%%%%%%%%%%%%%%%%%%%%%% EXTRAS
\section*{Puntos Extra}
\begin{enumerate}
  %%~~~~~~~~~~~~~~~~~~~~~~~~~~~~~~~~~~~~~~~~~~~~~~~~~~~ Extra 01
\item Sea $G$ una gr\'afica conexa en la que todo \'arbol de DFS es una
  trayectoria hamiltoniana (con la ra\'iz en uno de los extremos).   Demuestre
  que $G$ es un ciclo, una gr\'afica completa, o una gr\'afica bipartita completa
  en la que ambas partes tienen el mismo n\'umero de v\'ertices.
  
  %%~~~~~~~~~~~~~~~~~~~~~~~~~~~~~~~~~~~~~~~~~~~~~~~~~~~ Extra 02
\item Modifique BFS para que sea recursivo en lugar de iterativo.
  
  %%~~~~~~~~~~~~~~~~~~~~~~~~~~~~~~~~~~~~~~~~~~~~~~~~~~~ Extra 03
\item Modifique DFS para que sea recursivo en lugar de iterativo.
  
  %%~~~~~~~~~~~~~~~~~~~~~~~~~~~~~~~~~~~~~~~~~~~~~~~~~~~ Extra 04
\item Modifique al algoritmo BFS para que:
  \begin{enumerate}
    %%~~~~~~~~~~~~~~~~~~~~~~~~~~~~~~~~~~~~~~~~~~~~~~~~~~~ inciso (a)
  \item Reciba una gr\'afica no necesariamente conexa con dos
    v\'ertices distinguidos $r$ y $t$.
    %%~~~~~~~~~~~~~~~~~~~~~~~~~~~~~~~~~~~~~~~~~~~~~~~~~~~ inciso (b)
  \item El algoritmo empiece en $r$, y termine cuando encuentre al
    v\'ertice $t$, en cuyo caso lo regresa, junto con una trayectoria de
    longitud m\'inima de $r$ a $t$, o cuando decida que el v\'ertice $t$
    no puede ser alcanzado desde $r$, en cuyo caso regresa el valor
    \texttt{false}.
    %%~~~~~~~~~~~~~~~~~~~~~~~~~~~~~~~~~~~~~~~~~~~~~~~~~~~ inciso (c)
  \item El primer paso dentro del loop de \texttt{while} sea {\bf eliminar} la
    cabeza de la cola.
  \end{enumerate}

\end{enumerate}
\end{document}
