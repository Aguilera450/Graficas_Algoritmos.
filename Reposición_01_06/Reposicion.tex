\documentclass{article}

% Symbols
\usepackage{recycle}
\usepackage{amsfonts, amsthm}
\usepackage{upgreek}
\usepackage{physics}
\usepackage{cancel}
\usepackage{amssymb, latexsym, amsmath}

% Proof
\renewcommand*{\proofname}{\textbf{Demostraci\'on:}}

% Graphics
\usepackage{graphicx}
\usepackage{pgf}

% Color a letras.
%\usepackage[usenames,dvipsnames,svgnames,table]{xcolor}

% Tikz
\usepackage{tikz}
\usetikzlibrary{arrows,automata}
\usepackage{tikz}
\usetikzlibrary{arrows,automata}

\usetikzlibrary{shapes,calc}
\tikzstyle{edge}=[shorten <=2pt, shorten >=2pt,
  >=stealth, line width=1.1pt]
\tikzstyle{blueE}=[shorten <=2pt, shorten >=2pt,
  >=stealth, line width=1.5pt, blue]
\tikzstyle{blackV}=[circle, fill=black,
  minimum size=6pt,
  inner sep=0pt, outer sep=0pt]
\tikzstyle{blueV}=[circle, fill=blue, draw,
  minimum size=6pt, line width=0.75pt,
  inner sep=0pt, outer sep=0pt]
\tikzstyle{redV}=[circle, fill=red, draw,
  minimum size=6pt, line width=0.75pt,
  inner sep=0pt, outer sep=0pt]
\tikzstyle{redSV}=[semicircle, fill=red, minimum
  size=3pt, inner sep=0pt, outer sep=0pt,
  rotate=225]
\tikzstyle{blueSV}=[semicircle, fill=blue, minimum
  size=3pt, inner sep=0pt, outer sep=0pt,
  rotate=225]
\tikzstyle{blackSV}=[semicircle, fill=black, minimum
  size=3pt, inner sep=0pt, outer sep=0pt,
  rotate=225]
\tikzstyle{vertex}=[circle, draw, minimum size=6pt,
  line width=0.75pt, inner sep=0pt,
  outer sep=0pt]

% Margins
\addtolength{\voffset}{-1cm}
\addtolength{\hoffset}{-1.5cm}
\addtolength{\textwidth}{3cm}
\addtolength{\textheight}{2cm}

%Header-Footer
\usepackage{fancyhdr}
\renewcommand{\headrulewidth}{1pt}

\newcommand{\set}[1]{%
  \left\{ #1 \right\}%
}

\pagenumbering{gobble}
\footskip = 50pt
\renewcommand{\headrulewidth}{1pt}

\pagestyle{fancyplain}

\begin{document}
\title{UNIVERSIDAD AUT\'ONOMA DE M\'EXICO\\ Facultad de Ciencias}
\author{Autores:
  \\ Fernanda Villaf\'an Flores
  \\ Fernando Alvarado Palacios
  \\ Adri\'an Aguilera Moreno}
\date{}
\maketitle
\begin{center}
  \includegraphics[scale=0.20]{../Imagen/Portada.jpg}\\[0.4cm]
  \Large
  \bf{Gr\'aficas y Juegos}
  \normalsize
\end{center}
\newpage
\fancyhead[r]{ Gr\'aficas y Juegos 2022-1}
\section*{\LARGE{Reposioción}}
\begin{enumerate}
  %%%%%%%%%%%%%%%%%%%%%%%%%%%%%%%%%%%%%%%%%%%%%%%%%% Ejercicio 3 de la Tarea 02.
\item \text{[Ejercicio 3 de la Tarea 02]} Sea $G$ una gr\'afica conexa. Demuestre
  que si $G$ no es completa, entonces contiente a $P_3$ como subgr\'afica inducida.
    \renewcommand\qedsymbol{QED}
  \begin{proof} Para este ejercicio necesitamos que $|V_G| \geq 3$, para las
    gr\'aficas que no cumplan esto se tendr\'a la demostraci\'on por vacuidad.
    Nótese que el hecho de que $G$ no sea completa implica que para al
    menos $x,y \in V_G$ se tiene que $xy \notin E_G$.

        \begin{center}
      \fbox{
        \begin{minipage}[b][1\height]%
          [t]{0.867\textwidth}
          Previo a la demostraci\'on, provemos que en una gr\'afica conexa
          siempre podemos construir una trayectoria con exactamente $3$
          v\'ertices:
          \vspace*{0.3cm}

          Sea $x \in V_G$, por definici\'on de conexidad y como $|V_G| \geq 3$,
          tenemos ha $x,y \in V_G$ tales que $xy \in E_G$, luego $x$ es vecino
          a alg\'un v\'ertice distinto a $y$ (o $y$ es vecino de alg\'un v\'ertice
          distinto de $x$), pues en caso contrario $xy$ ser\'ia una componente
          conexa contenida en $G$ y $xy \not= G$!! lo que contradice la hip\'otesis
          de que $G$ es conexa. Supongamos, sin p\'erdida de generalidad, que $z$
          es vecino de $x$ y $z \not= y$, luego $zxy$ es una trayectoria de orden
          exactamente $3$. \hfill $\square$
      \end{minipage}}
    \end{center}

    Para este ejercicio basta analizar $2$ posibles casos\footnote{Se
      analizan los casos ``extremos'', pues los casos intermedios son combinaciones
      de estos.}:

    \textcolor{blue}{Caso 1:} Si $G + e$ es completa, donde $e = xy-$arista para
    $x,y \in V_G$. Por \textbf{\textit{Prop.}} $\mathbf{1.64}$ y por hipótesis
    sabemos que existe un $xy$-camino en $G$, luego por \textbf{\textit{Prop.}}
    $\mathbf{1.62}$ sabemos que hay, en particular, una $xy-$trayectoria en $G$,
    luego hay alguna $xy-$trayectoria de orden $3$ (esto lo sabemos gracias al
    resultado mostrado previamente) y supongamos, sin p\'erdida de generalidad,
    que \'esta es $T = (x,z,y)$, para $z \in V_G$, notemos que $T$ tiene tamaño
    igual a $2$, pues existen las aristas $zx, zy$ pero no $xy$ (por como definimos
    este caso), luego $T$ es $P_3$ y concluimos que $P_3$ es subgr\'afica inducida
    de $G$.

    \textcolor{blue}{Caso 2:} Si $G$ es un \'arbol, esto nos indica que $G$ es
    $1-$conexa, y es por eso que se considera este caso como el m\'inimo para el
    que se cumplir\'a la condici\'on a demostrar. Sabemos por el teorema de
    caracterizaci\'on de \'arboles que cada arista en $G$ ser\'a un puente, y
    por el resultado previamente mostrado sabemos que existe una trayectoria $T$
    en $G$ de orden exactamente $3$, as\'i $T$ es claramente $P_3$ y concluimos
    $P_3$ es subg\'afica inducida de $G$.

    De los casos anteriores concluimos que el enunciado es verdadero.
  \end{proof}

  %%%%%%%%%%%%%%%%%%%%%%%%%%%%%%%%%%%%%%%%%%%%%%%%%% Ejercicio 1 extra de la Tarea 02.
\item \text{[Ejercicio 1 extra de la Tarea 02]} Sea $G$ una gr\'afica. Demuestre que $G$
  es $k$-partita completa si y s\'olo si no contiene a $K_{k+1}$ ni a $\overline{P_3}$
  como subgr\'aficas inducidas.

  \begin{proof}
    En este ejercicio analizaremos 2 casos posibles:

    \begin{itemize}
    \item[$\Rightarrow$)] Procedamos reducción al absurdo .

      \begin{itemize}
      \item[$\cdot$)] Supongamos que $\overline{P_3}$ es subgr\'afica inducida
        de $G$, por definición de $k$-partita completa $\overline{P_3}$ no está
        en la misma parte (pues, en caso de estarlo hay una adyacencia en $2$
        vértices de la misma parte). Luego $\overline{P_3}$ está en $2$ o $3$
        partes distintas y habrá un $x \in \overline{P_3}$ que no se relacionará
        con al menos $1$ vértice en algunas de las partes y por tanto, $G$ no es
        $k$-partita completa!! (lo que no cumple es ser completa bajo el supuesto
        tomado) y he aquí una contradicción de suponer que $\overline{P_3}$ es
        subgr\'afica inducida de $G$. Por lo tanto, concluimos que $\overline{P_3}$
        no es subgr\'afica inducida de $G$.

      \item[$\cdot$)] Supongamos que $K_{k + 1}$ es subgr\'afica inducida de $G$,
        entonces hay $1$ vértice de $K_{k + 1}$ en cada una de las partes (lo que
        suma $k$ v\'ertices) y un $x \in K_{k + 1}$ en alguna parte tal que se
        relaciona con exactamente un v\'ertice en esa parte y por tanto, $G$ no
        es $k$-partita completa!! (no cumple el ser $k$-partita) y he aquí una
        contradicción de suponer que $K_{k + 1}$ es subgr\'afica inducida de $G$.
        Por lo tanto, concluimos que $K_{k + 1}$ no es subgr\'afica inducida de $G$.

      \item[$\Leftarrow$)]
      \end{itemize}
    \end{itemize}
    De los casos anterior concluimos que $G$ es $k$-partita completa si y
    s\'olo si no contiene a $K_{k+1}$ ni a $\overline{P_3}$ como subgr\'aficas
    inducidas.
  \end{proof}

  %%%%%%%%%%%%%%%%%%%%%%%%%%%%%%%%%%%%%%%%%%%%%%%%%% Ejercicio ...
\item
  %%%%%%%%%%%%%%%%%%%%%%%%%%%%%%%%%%%%%%%%%%%%%%%%%% Ejercicio ...
\item
  %%%%%%%%%%%%%%%%%%%%%%%%%%%%%%%%%%%%%%%%%%%%%%%%%% Ejercicio ...
\item
  %%%%%%%%%%%%%%%%%%%%%%%%%%%%%%%%%%%%%%%%%%%%%%%%%% Ejercicio ...
\item
  %%%%%%%%%%%%%%%%%%%%%%%%%%%%%%%%%%%%%%%%%%%%%%%%%% Ejercicio ...
\item
  %%%%%%%%%%%%%%%%%%%%%%%%%%%%%%%%%%%%%%%%%%%%%%%%%% Ejercicio ...
\item
  %%%%%%%%%%%%%%%%%%%%%%%%%%%%%%%%%%%%%%%%%%%%%%%%%% Ejercicio ...
\item
  %%%%%%%%%%%%%%%%%%%%%%%%%%%%%%%%%%%%%%%%%%%%%%%%%% Ejercicio ...
\item
\end{enumerate}
\end{document}
