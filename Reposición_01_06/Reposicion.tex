\documentclass{article}

% Symbols
\usepackage{recycle}
\usepackage{amsfonts, amsthm}
\usepackage{upgreek}
\usepackage{physics}
\usepackage{cancel}
\usepackage{amssymb, latexsym, amsmath}

% Proof
\renewcommand*{\proofname}{\textbf{Demostraci\'on:}}

% Graphics
\usepackage{graphicx}
\usepackage{pgf}

% Color a letras.
%
%\usepackage[usenames,dvipsnames,svgnames,table]{xcolor}

% Tikz
\usepackage{tikz}
\usetikzlibrary{arrows,automata}
\usepackage{tikz}
\usetikzlibrary{arrows,automata}

\usetikzlibrary{shapes,calc}
\tikzstyle{edge}=[shorten <=2pt, shorten >=2pt,
  >=stealth, line width=1.1pt]
\tikzstyle{blueE}=[shorten <=2pt, shorten >=2pt,
  >=stealth, line width=1.5pt, blue]
\tikzstyle{blackV}=[circle, fill=black,
  minimum size=6pt,
  inner sep=0pt, outer sep=0pt]
\tikzstyle{blueV}=[circle, fill=blue, draw,
  minimum size=6pt, line width=0.75pt,
  inner sep=0pt, outer sep=0pt]
\tikzstyle{redV}=[circle, fill=red, draw,
  minimum size=6pt, line width=0.75pt,
  inner sep=0pt, outer sep=0pt]
\tikzstyle{redSV}=[semicircle, fill=red, minimum
  size=3pt, inner sep=0pt, outer sep=0pt,
  rotate=225]
\tikzstyle{blueSV}=[semicircle, fill=blue, minimum
  size=3pt, inner sep=0pt, outer sep=0pt,
  rotate=225]
\tikzstyle{blackSV}=[semicircle, fill=black, minimum
  size=3pt, inner sep=0pt, outer sep=0pt,
  rotate=225]
\tikzstyle{vertex}=[circle, draw, minimum size=6pt,
  line width=0.75pt, inner sep=0pt,
  outer sep=0pt]

% Margins
\addtolength{\voffset}{-1cm}
\addtolength{\hoffset}{-1.5cm}
\addtolength{\textwidth}{3cm}
\addtolength{\textheight}{2cm}

%Header-Footer
\usepackage{fancyhdr}
\renewcommand{\headrulewidth}{1pt}

\newcommand{\set}[1]{%
  \left\{ #1 \right\}%
}

\pagenumbering{gobble}
\footskip = 50pt
\renewcommand{\headrulewidth}{1pt}

\pagestyle{fancyplain}

\begin{document}
\title{UNIVERSIDAD AUT\'ONOMA DE M\'EXICO\\ Facultad de Ciencias}
\author{Autores:
  \\ Fernanda Villaf\'an Flores
  \\ Fernando Alvarado Palacios
  \\ Adri\'an Aguilera Moreno}
\date{}
\maketitle
\begin{center}
  \includegraphics[scale=0.20]{../Imagen/Portada.jpg}\\[0.4cm]
  \Large
  \bf{Gr\'aficas y Juegos}
  \normalsize
\end{center}
\newpage
\fancyhead[r]{ Gr\'aficas y Juegos 2022-1}
\section*{\LARGE{Reposioción}}
\begin{enumerate}
  %%%%%%%%%%%%%%%%%%%%%%%%%%%%%%%%%%%%%%%%%%%%%%%%%% Ejercicio 3 de la Tarea 02.
\item \text{[Ejercicio 3 de la Tarea 02]} Sea $G$ una gr\'afica conexa. Demuestre
  que si $G$ no es completa, entonces contiente a $P_3$ como subgr\'afica inducida.
    \renewcommand\qedsymbol{QED}
  \begin{proof} Para este ejercicio necesitamos que $|V_G| \geq 3$, para las
    gr\'aficas que no cumplan esto se tendr\'a la demostraci\'on por vacuidad.
    Nótese que el hecho de que $G$ no sea completa implica que para al
    menos $x,y \in V_G$ se tiene que $xy \notin E_G$.

        \begin{center}
      \fbox{
        \begin{minipage}[b][1\height]%
          [t]{0.867\textwidth}
          Previo a la demostraci\'on, provemos que en una gr\'afica conexa
          siempre podemos construir una trayectoria con exactamente $3$
          v\'ertices:
          \vspace*{0.3cm}
          
          Sea $x \in V_G$, por definici\'on de conexidad y como $|V_G| \geq 3$,
          tenemos ha $x,y \in V_G$ tales que $xy \in E_G$, luego $x$ es vecino
          a alg\'un v\'ertice distinto a $y$ (o $y$ es vecino de alg\'un v\'ertice
          distinto de $x$), pues en caso contrario $xy$ ser\'ia una componente
          conexa contenida en $G$ y $xy \not= G$!! lo que contradice la hip\'otesis
          de que $G$ es conexa. Supongamos, sin p\'erdida de generalidad, que $z$
          es vecino de $x$ y $z \not= y$, luego $zxy$ es una trayectoria de orden
          exactamente $3$. \hfill $\square$
      \end{minipage}}
    \end{center}
    
    Para este ejercicio basta analizar $2$ posibles casos\footnote{Se
      analizan los casos ``extremos'', pues los casos intermedios son combinaciones
      de estos.}:
    
    \textcolor{blue}{Caso 1:} Si $G + e$ es completa, donde $e = xy-$arista para
    $x,y \in V_G$. Por \textbf{\textit{Prop.}} $\mathbf{1.64}$ y por hipótesis
    sabemos que existe un $xy$-camino en $G$, luego por \textbf{\textit{Prop.}}
    $\mathbf{1.62}$ sabemos que hay, en particular, una $xy-$trayectoria en $G$,
    luego hay alguna $xy-$trayectoria de orden $3$ (esto lo sabemos gracias al
    resultado mostrado previamente) y supongamos, sin p\'erdida de generalidad,
    que \'esta es $T = (x,z,y)$, para $z \in V_G$, notemos que $T$ tiene tamaño
    igual a $2$, pues existen las aristas $zx, zy$ pero no $xy$ (por como definimos
    este caso), luego $T$ es $P_3$ y concluimos que $P_3$ es subgr\'afica inducida
    de $G$.
    
    \textcolor{blue}{Caso 2:} Si $G$ es un \'arbol, esto nos indica que $G$ es
    $1-$conexa, y es por eso que se considera este caso como el m\'inimo para el
    que se cumplir\'a la condici\'on a demostrar. Sabemos por el teorema de
    caracterizaci\'on de \'arboles que cada arista en $G$ ser\'a un puente, y
    por el resultado previamente mostrado sabemos que existe una trayectoria $T$
    en $G$ de orden exactamente $3$, as\'i $T$ es claramente $P_3$ y concluimos
    $P_3$ es subg\'afica inducida de $G$.
    
    De los casos anteriores concluimos que el enunciado es verdadero.
  \end{proof}
  
  %%%%%%%%%%%%%%%%%%%%%%%%%%%%%%%%%%%%%%%%%%%%%%%%%% Ejercicio 1 extra de la Tarea 02.
\item \text{[Ejercicio 1 extra de la Tarea 02]} Sea $G$ una gr\'afica. Demuestre que $G$
  es $k$-partita completa si y s\'olo si no contiene a $K_{k+1}$ ni a $\overline{P_3}$
  como subgr\'aficas inducidas.
  
  \begin{proof}
    En este ejercicio analizaremos 2 casos posibles:

    \begin{itemize}
    \item[$\Rightarrow$)] Procedamos reducción al absurdo .

      \begin{itemize}
      \item[$\cdot$)] Supongamos que $\overline{P_3}$ es subgr\'afica inducida
        de $G$, por definición de $k$-partita completa $\overline{P_3}$ no está
        en la misma parte (pues, en caso de estarlo hay una adyacencia en $2$
        vértices de la misma parte). Luego $\overline{P_3}$ está en $2$ o $3$
        partes distintas y habrá un $x \in \overline{P_3}$ que no se relacionará
        con al menos $1$ vértice en algunas de las partes y por tanto, $G$ no es
        $k$-partita completa!! (lo que no cumple es ser completa bajo el supuesto
        tomado) y he aquí una contradicción de suponer que $\overline{P_3}$ es
        subgr\'afica inducida de $G$. Por lo tanto, concluimos que $\overline{P_3}$
        no es subgr\'afica inducida de $G$.
        
      \item[$\cdot$)] Supongamos que $K_{k + 1}$ es subgr\'afica inducida de $G$,
        entonces hay $1$ vértice de $K_{k + 1}$ en cada una de las partes (lo que
        suma $k$ v\'ertices) y un $x \in K_{k + 1}$ en alguna parte tal que se
        relaciona con exactamente un v\'ertice en esa parte y por tanto, $G$ no
        es $k$-partita completa!! (no cumple el ser $k$-partita) y he aquí una
        contradicción de suponer que $K_{k + 1}$ es subgr\'afica inducida de $G$.
        Por lo tanto, concluimos que $K_{k + 1}$ no es subgr\'afica inducida de $G$.
        
      \item[$\Leftarrow$)] 
      \end{itemize}
    \end{itemize}
    De los casos anterior concluimos que $G$ es $k$-partita completa si y
    s\'olo si no contiene a $K_{k+1}$ ni a $\overline{P_3}$ como subgr\'aficas
    inducidas.
  \end{proof}

  %%%%%%%%%%%%%%%%%%%%%%%%%%%%%%%%%%%%%%%%%%%%%%%%%% Parte Fer jajajjaaj ...
  %%%%%%%%%%%%%%%%%%%%%%%%%%%%%%%%%%%%%%%%%%%%%%%%%% Ejercicio 3 de la tarea 4
  Ejercicio de la Tarea 4

  $\blacktriangleright$ Ejercicio extra 3
  
   \item Sea $\mathcal{T}$ una familia de sub\'arboles
      de un \'arbol $T$.   Deduzca, por inducci\'on sobre
      $|\mathcal{T}|$, que si cualesquiera dos elementos
      de $\mathcal{T}$ tienen un v\'ertice en com\'un,
      entonces hay un v\'ertice de $T$ que est\'a en
      todos los elementos de $\mathcal{T}$.
      
  \begin{proof} 
      Demostración por induccion sobre el numero de subarboles
      
  Paso base: $\mathcal{T} = 3$
  
  Sean $T_1$, $T_2$, $T_3$ subarboles de T 
  
  Pd Existe  un vértice x que pertenece a T tal que $T_1 \cap T_2=x \rightarrow  T_1 \cap T_2 \cap T_3 = x$ 
  
  Dem (Reduccion al absurdo)
  
  Supongamos que $T_1 \cap T_2 \cap T_3 = \emptyset$ (algo que no puede pasar es que $T_1 \cap T_2 = T_1 \cap T_3 = T_2 \cap T_3  = \emptyset$ por hipotesis T es conexa y tambien por hipotesis $T_1 \cap T_2$ diferente del vacio) $\rightarrow$ existe $x_1$ tal que $x_1$ pertenece a $T_1 \cap T_2$ y $x_1$ no pertenece a $T_3$, existe $x_2$ tal que $x_2$ pertenece a $T_1 \cap T_3$ y $x_2$ no pertenece a $T_2$, existe $x_3$ tal que $x_3$ pertenece a $T_2 \cap T_3$ y $x_3$ no pertenece a $T_1  \rightarrow x_1, x_2, x_3$ son vértices diferentes $\rightarrow$ por las intersecciones  y gracias que T  es conexo  $\rightarrow$ s.p.g. podemos formar un ciclo $(x_1, ... ,x_2, ... ,x_3, ...,x_1)$ (Lo que es una contradicción, ya que los arboles son aciclicos)
  
  Por lo tanto $T_1 \cap T_2 \cap T_3   \neq
   \emptyset \rightarrow$ existe un vértice compartido para los 3 subárboles
   
   Hipótesis de inducción: supongamos para $\mathcal{T} = k$
   
   Supongamos que si $\mathcal{T} = k$ tal que existen i,j tal que $i \neq j$ e i,j pertenezen a $\left\lbrace 1,...,k \right\rbrace$ donde $T_i \cap T_j \neq \emptyset \rightarrow  \bigcap_{r=1}^{k}{T_r} \neq \emptyset$
   
   Paso inductivo: Pd para $\mathcal{T} = k+1$
   
   Pd si existen i,j tal que $i \neq j$ e i,j pertenezen a $\left\lbrace 1,...,k+1 \right\rbrace$ donde $T_i \cap T_j \neq \emptyset \rightarrow  \bigcap_{r=1}^{k+1}{T_r} \neq \emptyset$
   
   Dem (Reducción al absurdo)
   
   Supongamos $\bigcap_{r=1}^{k+1}{T_r} = \emptyset$ y existen i,j pertenezen a $\left\lbrace 1,...,k+1 \right\rbrace$ donde $T_i \cap T_j \neq \emptyset \rightarrow $ consideremos a T' como el subárbol formado por la unión de todos los subárboles de $T_1, T_2,...;T_{k+1}$ menos los subárboles $T_i$ y $T_j$. Es decir T'= $\bigcup_{r=1, r \neq i,  r \neq j }^{k+1}{T_r} \rightarrow T' \cap T_i \cap T_j = \emptyset$, pero por paso base esto es una contradicción $\rightarrow$ existe un x tal que x pertenece a  $\bigcap_{r=1}^{k+1}{T_r}$ 
   
   Por lo tanto la porposición es verdadera.
  \end{proof}
  

  %%%%%%%%%%%%%%%%%%%%%%%%%%%%%%%%%%%%%%%%%%%%%%%%%% Tarea 5
  Ejercicios de la tarea 5 
  

  %%%%%%%%%%%%%%%%%%%%%%%%%%%%%%%%%%%%%%%%%%%%%%%%%% Preguta 1
  
  $\blacktriangleright$ Pregunta 1 
  
  \item Demuestre que si $G$ es simple y $3$-regular, entonces $\kappa = \kappa'$.
  
  \begin{proof} 
  
  Sea G 3-regular $\rightarrow d(v)=3$ para todo v que pertenece a V $\rightarrow$ sea $v'$ un vertice de G  $\rightarrow$ para desconectar a $v'$ de G, solo basta  con "cortar" las 3 aristas de $v' \rightarrow$ k'=3
  
  Por lo tanto $k=k'$
  \end{proof}
  

%%%%%%%%%%%%%%%%%%%%%%%%%%%%%%%%%%%%%%%%%%%%%%%%%% Pregunta 2

  $\blacktriangleright$ Pregunta 2
  
  \item Demuestre que una gr\'afica es $2$-conexa por aristas si y s\'olo si
    cualesquiera dos v\'ertices est\'an conectados por al menos dos trayectorias
    ajenas por aristas.
  
  \begin{proof} 
     $\Longrightarrow$) Sea G una gráfica 2-conexa $\rightarrow$ por teorema visto en clase en G existe  un cliclo C que contendra 2 vertices v y u donde u y v pertenecen a G $\rightarrow$ podemos tener la trayectoria $P=(v,C,u)$ pero como C es un ciclo $\rightarrow$ tambien existira la trayectoria $P'=(u,V,v)$ 
     
     Por lo tanto existen 2 trayectorias ajenas por aristas en una gráfica 2-conexa 
     
     
     $\Longleftarrow$) Sean u y v cualesquiera vértices de una gráfica G y si u y v están conectados por dos trayectorias ajenas por aristas P y P' $\rightarrow$ si unimos  uPv y vP'v obtendremos un ciclo C que ira de uPvP'u $\rightarrow$ sea G' una gráfica igual al ciclo C, si borramos una arista a G' $\rightarrow$ G' seguira siendo conexa  $\rightarrow$ G' es 2-conexa $\rightarrow$ G será 2-conexa ya que para todo v y u que pertenecen G existen 2 trayectorias ajenas por vertices
     
      
  \end{proof}
  
  %%%%%%%%%%%%%%%%%%%%%%%%%%%%%%%%%%%%%%%%%%%%%%%%%% Ejercicio ...
\item
  %%%%%%%%%%%%%%%%%%%%%%%%%%%%%%%%%%%%%%%%%%%%%%%%%% Ejercicio ...
\item
  %%%%%%%%%%%%%%%%%%%%%%%%%%%%%%%%%%%%%%%%%%%%%%%%%% Ejercicio ...
\item
  %%%%%%%%%%%%%%%%%%%%%%%%%%%%%%%%%%%%%%%%%%%%%%%%%% Ejercicio ...
\item
  %%%%%%%%%%%%%%%%%%%%%%%%%%%%%%%%%%%%%%%%%%%%%%%%%% Ejercicio ...
\item
  %%%%%%%%%%%%%%%%%%%%%%%%%%%%%%%%%%%%%%%%%%%%%%%%%% Ejercicio ...
\item
  %%%%%%%%%%%%%%%%%%%%%%%%%%%%%%%%%%%%%%%%%%%%%%%%%% Ejercicio ...
\item
\end{enumerate}
\end{document}
