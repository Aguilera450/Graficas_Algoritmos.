\documentclass{article}

% Symbols
%\usepackage{recycle}
\usepackage{amsfonts, amsthm}
\usepackage{upgreek}
\usepackage{physics}
\usepackage{cancel}
\usepackage{amssymb, latexsym, amsmath}

% Proof
\renewcommand*{\proofname}{\textbf{Demostraci\'on:}}

% Graphics
\usepackage{graphicx}
\usepackage{pgf}

% Color a letras.
%\usepackage[usenames,dvipsnames,svgnames,table]{xcolor}

% Tikz
\usepackage{tikz}
\usetikzlibrary{arrows,automata}
\usepackage{tikz}
\usetikzlibrary{arrows,automata}

\usetikzlibrary{shapes,calc}
\tikzstyle{edge}=[shorten <=2pt, shorten >=2pt,
  >=stealth, line width=1.1pt]
\tikzstyle{blueE}=[shorten <=2pt, shorten >=2pt,
  >=stealth, line width=1.5pt, blue]
\tikzstyle{blackV}=[circle, fill=black,
  minimum size=6pt,
  inner sep=0pt, outer sep=0pt]
\tikzstyle{blueV}=[circle, fill=blue, draw,
  minimum size=6pt, line width=0.75pt,
  inner sep=0pt, outer sep=0pt]
\tikzstyle{redV}=[circle, fill=red, draw,
  minimum size=6pt, line width=0.75pt,
  inner sep=0pt, outer sep=0pt]
\tikzstyle{redSV}=[semicircle, fill=red, minimum
  size=3pt, inner sep=0pt, outer sep=0pt,
  rotate=225]
\tikzstyle{blueSV}=[semicircle, fill=blue, minimum
  size=3pt, inner sep=0pt, outer sep=0pt,
  rotate=225]
\tikzstyle{blackSV}=[semicircle, fill=black, minimum
  size=3pt, inner sep=0pt, outer sep=0pt,
  rotate=225]
\tikzstyle{vertex}=[circle, draw, minimum size=6pt,
  line width=0.75pt, inner sep=0pt,
  outer sep=0pt]

% Margins
\addtolength{\voffset}{-0.5cm}
\addtolength{\hoffset}{-0.5cm}
\addtolength{\textwidth}{1cm}
\addtolength{\textheight}{1cm}

%Header-Footer
\usepackage{fancyhdr}
\renewcommand{\headrulewidth}{1pt}

\newcommand{\set}[1]{%
  \left\{ #1 \right\}%
}

%\pagenumbering{gobble} -- Este comando
%                       -- quita el número de página.
\footskip = 50pt
\renewcommand{\headrulewidth}{1pt}

\pagestyle{fancyplain}

\begin{document}
\title{UNIVERSIDAD AUT\'ONOMA DE M\'EXICO\\ Facultad de Ciencias}
\author{Autores:
  \\ Fernanda Villaf\'an Flores
  \\ Fernando Alvarado Palacios
  \\ Adri\'an Aguilera Moreno}
\date{}
\maketitle
\begin{center}
  \includegraphics[scale=0.20]{../Imagen/Portada.jpg}\\[0.4cm]
  \Large
  \bf{Gr\'aficas y Juegos}
  \normalsize
\end{center}
\newpage
\fancyhead[r]{ Gr\'aficas y Juegos 2022-1}
\section*{\LARGE{Tarea 2}}


\begin{enumerate}
\item Demuestre que toda flecha en un camino dirigido cerrado en una
  digr\'afica pertenece a alg\'un ciclo dirigido.

\item Demuestre que si $G$ es simple y $\delta \ge 2$, entonces $G$ contiene
  un ciclo de longitud al menos $\delta + 1$.

\item Sea $G$ una gr\'afica conexa.   Demuestre que si $G$ no es completa,
  entonces contiente a $P_3$ como subgr\'afica inducida.

\item Demuestre que cualesquiera dos trayectorias de longitud m\'axima en una
  gr\'afica conexa tienen un vértice en común.

\item Caracterice a las gr\'aficas $k$-regulares para $k \in \{ 0, 1, 2 \}$.

\item Demuestre que si $|E| \ge |V|$, entonces $G$ contiene un ciclo.
\end{enumerate}

\section*{Puntos extra}

\begin{enumerate}
\item Sea $G$ una gr\'afica.   Demuestre que $G$ es $k$-partita completa si y
  s\'olo si no contiene a $K_{k+1}$ ni a $\overline{P_3}$ como subgr\'aficas
  inducidas.

\item Demuestre que si $G$ es una gr\'afica con $|V| \ge 4$ y $|E| > n^2/4$,
  entonces $G$ contiene un ciclo impar.

\item Sea $d = (d_1, \dots, d_n)$ una sucesi\'on no creciente de enteros no
  negativos. Sea $d' = (d_2-1, \dots, d_{d_1+1}-1, d_{d_1+2}, \dots, d_n)$.
  \begin{enumerate}
  \item Demuestre que $d$ es gr\'afica si y s\'olo si $d'$ es gr\'afica.

  \item Usando el primer inciso, describa un algoritmo que acepte como
    entrada una sucesi\'on no creciente de enteros no negativos $d$ y
    devuelva una gr\'afica simple con sucesi\'on de grados $d$, un
    certificado de que $d$ no es gr\'afica.
  \end{enumerate}
\end{enumerate}

\end{document}
