\documentclass{article}

% Symbols
%\usepackage{recycle}
\usepackage{amsfonts, amsthm}
\usepackage{upgreek}
\usepackage{physics}
\usepackage{cancel}
\usepackage{amssymb, latexsym, amsmath}

% Proof
\renewcommand*{\proofname}{\textbf{Demostraci\'on:}}

% Graphics
\usepackage{graphicx}
\usepackage{pgf}

% Color a letras.
%\usepackage[usenames,dvipsnames,svgnames,table]{xcolor}

% Tikz
\usepackage{tikz}
\usetikzlibrary{arrows,automata}
\usepackage{tikz}
\usetikzlibrary{arrows,automata}

\usetikzlibrary{shapes,calc}
\tikzstyle{edge}=[shorten <=2pt, shorten >=2pt,
  >=stealth, line width=1.1pt]
\tikzstyle{blueE}=[shorten <=2pt, shorten >=2pt,
  >=stealth, line width=1.5pt, blue]
\tikzstyle{blackV}=[circle, fill=black,
  minimum size=6pt,
  inner sep=0pt, outer sep=0pt]
\tikzstyle{blueV}=[circle, fill=blue, draw,
  minimum size=6pt, line width=0.75pt,
  inner sep=0pt, outer sep=0pt]
\tikzstyle{redV}=[circle, fill=red, draw,
  minimum size=6pt, line width=0.75pt,
  inner sep=0pt, outer sep=0pt]
\tikzstyle{redSV}=[semicircle, fill=red, minimum
  size=3pt, inner sep=0pt, outer sep=0pt,
  rotate=225]
\tikzstyle{blueSV}=[semicircle, fill=blue, minimum
  size=3pt, inner sep=0pt, outer sep=0pt,
  rotate=225]
\tikzstyle{blackSV}=[semicircle, fill=black, minimum
  size=3pt, inner sep=0pt, outer sep=0pt,
  rotate=225]
\tikzstyle{vertex}=[circle, draw, minimum size=6pt,
  line width=0.75pt, inner sep=0pt,
  outer sep=0pt]

% Margins
\addtolength{\voffset}{-0.5cm}
\addtolength{\hoffset}{-0.5cm}
\addtolength{\textwidth}{1cm}
\addtolength{\textheight}{1cm}

%Header-Footer
\usepackage{fancyhdr}
\renewcommand{\headrulewidth}{1pt}

\newcommand{\set}[1]{%
  \left\{ #1 \right\}%
}

%\pagenumbering{gobble} -- Este comando
%                       -- quita el número de página.
\footskip = 50pt
\renewcommand{\headrulewidth}{1pt}

\pagestyle{fancyplain}

\begin{document}
  \title{UNIVERSIDAD AUT\'ONOMA DE M\'EXICO\\ Facultad de Ciencias}
  \author{Autores:
    \\ Fernanda Villaf\'an Flores
    \\ Fernando Alvarado Palacios
    \\ Adri\'an Aguilera Moreno}
  \date{}
  \maketitle
  \begin{center}
    \includegraphics[scale=0.20]{../Imagen/Portada.jpg}\\[0.4cm]
    \Large
    \bf{Gr\'aficas y Juegos}
    \normalsize
  \end{center}
  \newpage
  \fancyhead[r]{ Gr\'aficas y Juegos 2022-1}
  \section*{\LARGE{Tarea 2}}

  \begin{enumerate}
    \small
    %%%%%%%%%%%%%%%%%%%%%%%%%%%%%%%% Ejercicio 1 %%%%%%%%%%%%%%%%%%%%%%%%%%%%%%%%
    \item Demuestre que toda flecha en un camino dirigido cerrado en una digr\'afica
      pertenece a alg\'un ciclo dirigido.

      \renewcommand\qedsymbol{QED}
      \begin{proof}
        Sea $C$ un camino dirigido cerrado tal que $C = (v_{0}, v_{1}, \dots, v_{i},
        \dots, v_{0})$.

        Procedamos por inducción sobre el número de veces que se repite un vértice
        distinto a $v_{0}$.

          \begin{itemize}
            \item \textbf{Paso Base:} $i = 0$.

            En este caso tenemos que el vértice $v_{i}$ no se repite ninguna vez, por
            lo que ya tenemos el ciclo dirigido buscado.

            \item \textbf{Hipótesis de Inducción:} Supongamos que $i > 0$ tal que haya un ciclo dirigido.

            \item \textbf{Paso Inductivo:} Demostraremos que si el vértice $v_{i}$ se
              repite al menos una vez, hay un ciclo dirigido.

              Como el vértice $v_{i}$ se repite, podemos dividir a $C$ de la siguiente forma:

              Sea $C'$ un camino dirigido cerrado de $C$. \\
              Tenemos que:
              $$C' = (v_{i}, \dots, v_{i})$$
              Entonces, si $C'$ no repite vértices, ya tenemos el ciclo dirigido buscado. \\
              En caso contrario, aplicamos el mismo proceso:

              Sea $C''$ un camino dirigido cerrado de $C'$. \\
              Tenemos que:
              $$C'' = (v_{0}, \dots, v_{i}, v_{j}, \dots, v_{0}) \text{ , con $i < j$}$$
              , donde $v_{0}$ es un vértice particular cualquiera en el camino dirigido $C$, el cuál tomamos
              como vértice inicial para el camino $C''$.

              Luego, este proceso lo realizamos hasta obtener un camino dirigido cerrado
              sin vértices repetidos.
          \end{itemize}
        Así, podemos concluir que toda flecha en un camino dirigido cerrado en una digráfica
        pertenece a algún ciclo dirigido.
      \end{proof}

    %%%%%%%%%%%%%%%%%%%%%%%%%%%%%%%% Ejercicio 2 %%%%%%%%%%%%%%%%%%%%%%%%%%%%%%%%
    \item Demuestre que si $G$ es simple y $\delta \ge 2$, entonces $G$ contiene
    un ciclo de longitud al menos $\delta + 1$.

      \renewcommand\qedsymbol{QED}
      \begin{proof}
        Procedemos por reducción a lo absurdo.

        Sea $P$ una trayectoria de longitud máxima en $G$ tal que $P = (v_{0}, \dots, v_{n})$. \\
        Por una \textbf{Proposición} vista en clase, sabemos que $G$ contiene un ciclo (digamos $C$). \\
        Supongamos que $C$ es de longitud máxima a lo más $\delta$, es decir, no existe un ciclo $C$
        que sea de longitud $\delta + 1$ o mayor. \\
        Si tomamos un vértice $v_{i}$ en $C$, sabemos que tiene al menos $\delta$ vecinos (de los cuales
        $\delta - 1$ vecinos pueden estar en $C$). Ahora, si $v_{i}$ es
        adyacente a otro vértice $w$ fuera del ciclo, tenemos dos casos:
        \begin{itemize}
          \item El vértice $v_{i-1}$ también es adyacente a $w$.

            En este caso, ya tendríamos un ciclo de grado $\delta + 1$ !!! .

          \item El vértice $v_{i-1}$ no es adyacente a $w$.

            En este caso, $v_{i-1}$ continuaría su adyacencia por otros vértices. Si tomamos a $G$
            una gráfica conexa, en algún vértice $v_{j}$, llegaríamos a que $v_{j}$ es adyacente a $w$. \\
            Por lo tanto, ya tendríamos un ciclo de longitud mayor a $\delta$ (al menos $\delta + 1$)!!! .
        \end{itemize}

        La contradicción surge de suponer que la longitud máxima de un ciclo $C$ es a lo más $\delta$. \\
        Por lo tanto, su longitud máxima es de al menos $\delta + 1$.
      \end{proof}

    %%%%%%%%%%%%%%%%%%%%%%%%%%%%%%%% Ejercicio 3 %%%%%%%%%%%%%%%%%%%%%%%%%%%%%%%%
    \item Sea $G$ una gr\'afica conexa. Demuestre que si $G$ no es completa,
      entonces contiente a $P_3$ como subgr\'afica inducida.

      \renewcommand\qedsymbol{QED}
      \begin{proof}
        Procedamos por reducción al absurdo.

        Sea $G$ una gráfica completa, entonces $P_{3}$ es subgráfica de $G$. \\
        Para este ejercicio necesitamos de una condición, $V_G \ge 3$, para las gráficas que no
        cumplan esto se tendrá la demostración por vacuidad.

        Tomemos a $x_{i - 1}, x_{i}, x_{i + 1}$ en $V_{G}$ ($2 \ge i \ge
        |V_{G}| - 1$). Como $G$ es completa, se tiene que la distancia entre cualesquiera
        $2$ v\'ertices es $1$. Luego, tenemos que hay ${3 \choose 2}$ aristas!! y esto es
        claramente mayor que $2$ ($|E_{P_3}|$). Como los v\'ertices que tomamos son arbitrarios,
        podemos concluir que $P_{3} \nsubseteq G$.

        Como la anterior contradicci\'on resulta de suponer a $G$ completa, podemos
        asegurar que si $G$ no es completa, entonces $P_3 \subseteq G$.
      \end{proof}

    %%%%%%%%%%%%%%%%%%%%%%%%%%%%%%%% Ejercicio 4 %%%%%%%%%%%%%%%%%%%%%%%%%%%%%%%%
    \item Demuestre que cualesquiera dos trayectorias de longitud m\'axima en una
      gr\'afica conexa tienen un vértice en común.

      \renewcommand\qedsymbol{QED}
      \begin{proof}
        Veamos los siguientes casos:

        \begin{itemize}
          \item Si $G$ no tiene vértices de corte.

            Como $c(G) = 1$, entonces hay una trayectoria que pasa por todos los vértices
            y a su vez es de longitud máxima. \\
            Por tanto, ya terminamos.

          \item Si $G$ tiene vértices de corte.

            Analicemos el caso extremo, que engloba a todos los posibles casos con al menos
            un vértice de corte. \\
            Sean $T_{1} = uv$-trayectoria y $T_{2} = xy$-trayectoria, ambas de longitud
            máxima y a su vez, ajenas por aristas entre sí. Luego, como $G$ es conexa existe
            un $ux$-camino $C$ ($uy, vx, vy$ caminos) y como parte de él hay un vértice $a$
            tal que es el vértice más próximo a $x$ que está en $T_{1}$ (y es el único con esta
            propiedad). \\
            Si $a$ forma parte de $T_{2}$, ya terminamos. \\
            En caso contrario, entonces $a$ puede ser de corte o no. \\
            Así:

            \begin{itemize}
              \item Si $a$ es de corte, entonces de $xCa$ será el mínimo camino donde $b$ es el
                vértice más cercano a $u$ que se encuentra en $T_{2}$ y es el único con esta propiedad. \\
                Por lo que hay una $bx$-trayectoria y $by$-trayectoria que si las comparamos y tomamos la más
                longeva (llamémosla $T_{3}$), de tener la misma longitud es indiferente cuál tomemos. \\
                Así, tomamos un $ab$-camino sin ciclos, \textit{i.e.} $ab$-trayectoria. \\
                Luego, llamamos $T_{4}$ a la trayectoria que resulte de mayor longitud entre $au$-trayectoria
                y $av$-trayectoria. Si son iguales, nuevamente es indiferente cuál tomemos. \\
                De lo anterior, nótese que:
                \begin{eqnarray*}
                  \mathcal{L}(T_3) &\geq& \frac{\mathcal{L}(T_1)}{2} = \frac{\mathcal{L}(T_2)}{2}\\
                  \mathcal{L}(T_4) &\geq& \frac{\mathcal{L}(T_2)}{2} = \frac{\mathcal{L}(T_1)}{2}\\
                  \mathcal{L}(ab) &\geq& 1
                \end{eqnarray*}
                Así, hay una trayectoria $T_{5} = T_{3} ab T_{4}: \mathcal{L}(T_{5}) > \mathcal{L}(T_{1})$. \\
                Por lo que se contradice que $T_{1}$ y $T_{2}$ fueran de longitud máxima. La contradicción surge de
                suponer que $a$ es de corte y si está en $T_{1}$ no está en $T_{2}$. \\
                Por lo tanto, concluimos que $a$ no es de corte y no forma parte de $T_{2}$.

              \item Si $a$ no es de corte, entonces es parte de un extremo de $T_{1}$. \\
                Como $a$ está en $C$, tenemos que $T_{1}$ más la arista que tiene como vértice $a$ y
                no es parte de $T_{1}$ es una trayectoria más larga que $T_{1}$ y $T_{2}$. De aquí una tenemos una
                contradicción al suponer que $a$ no es de corte y además no está en $T_{2}$.
            \end{itemize}
        \end{itemize}
        Después del análisis anterior, observemos que llegamos a contradicciones que son resultado
        de suponer que: \\
        Si $a$ está en $T_{1}$, entonces no esta en $T_{2}$.

        Por lo tanto, se sigue que si $a$ esta en $T_{1}$ entonces está en $T_{2}$.
      \end{proof}

    %%%%%%%%%%%%%%%%%%%%%%%%%%%%%%%% Ejercicio 5 %%%%%%%%%%%%%%%%%%%%%%%%%%%%%%%%
    \item Caracterice a las gr\'aficas $k$-regulares para $k \in \{ 0, 1, 2 \}$.

      \textit{\textbf{Soluci\'on:}}
      \begin{itemize}
        \item[$k = 0$)] Son todas las gráficas que no tienen aristas, a estas se les conoce
          como gr\'aficas vac\'ias.

        \item[$k = 1$)] Estas son gr\'aficas con una cantidad de v\'ertices par y son tantas
          uniones de $P_2$ como $\frac{|V_G|}{2}$. Las gráficas con $|V_G|$ impar no entran
          aqu\'i porque siempre habr\'a $(|V_G| - 1) P_2$ y alg\'un v\'ertice (aislado) será
          de grado igual a $0$!! (pensando en lazos, de grado igual a 2). Esto contradice
          el ser $1-$regular.

        \item[$k = 2$)] Son gr\'aficas que contienen ciclos o son combinaciones de ciclos.
          Todos los ciclos son $2-$regulares y esto no implica que todas las gr\'aficas
          $2-$regulares sean un ciclo, pero si que los contengan o que sean combinaciones
          de estos.
      \end{itemize}
      Con los $3$ puntos anteriores concluimos la caracterización. \hfill $\square$

    %%%%%%%%%%%%%%%%%%%%%%%%%%%%%%%% Ejercicio 6 %%%%%%%%%%%%%%%%%%%%%%%%%%%%%%%%
    \item Demuestre que si $|E| \ge |V|$, entonces $G$ contiene un ciclo.

      \renewcommand\qedsymbol{QED}
      \begin{proof} (Por contradicción)

        Supongamos que $G$ es una gráfica tal que $|E|\geq|V|$ y $G$ no contiene ningún ciclo. \\
        Entonces, el número máximo de vértices de este tipo de gráficas será igual a $|V|-1$, ya que
        todo vértice se podría relacionar con algún otro vértice sin repetir con los vértices anteriores.
        Pero el último no se podrá relacionar con algún otro vértice ya que si lo hiciera, formaría
        un ciclo. \\
        Por lo que, $|E|\leq|V|-1 < |V|$ !!! (lo que es una contradicción, pues contradice nuestra hipótesis
        de que $|E|\geq|V|$).

        Por lo tanto, la gráfica $G$ debe tener al menos un ciclo.
        \end{proof}
  \end{enumerate}

  \section*{Puntos extra}

  \begin{enumerate}
    %%%%%%%%%%%%%%%%%%%%%%%%%%%%%%%% Ejercicio 1 %%%%%%%%%%%%%%%%%%%%%%%%%%%%%%%%
    \item Sea $G$ una gr\'afica. Demuestre que $G$ es $k$-partita completa si y
      s\'olo si no contiene a $K_{k+1}$ ni a $\overline{P_3}$ como subgr\'aficas
      inducidas.

      \renewcommand\qedsymbol{QED}
      \begin{proof}
        En este ejercicio analizaremos 2 casos posibles:

        \begin{itemize}
        \item[$\Rightarrow$)] Procedamos por contrapositiva.

          \begin{itemize}
            \item[$\cdot$)] Si $\overline{P_3} \subseteq G$, por definición
              de $k$-partita completa $\overline{P_3}$ no está en la misma
              parte (ya que en caso de estarlo, hay una adyacencia en $2$ vértices
              de la misma parte). Luego, $\overline{P_3}$ está en $2$ o $3$
              partes distintas y habrá un $x \in \overline{P_3}$ que no se
              relacionará con al menos $1$ vértice en algunas de las partes
              y por tanto, $G$ no es $k$-partita completa (lo que no cumple es
              ser completa bajo el supuesto tomado).

            \item[$\cdot$)] Si $K_{k + 1} \subseteq G$, entonces hay $1$ vértice
              de $K_{k + 1}$ en cada una de las partes (lo que suma $k$ v\'ertices) y un
              $x \in K_{k + 1}$ en alguna parte tal que se relaciona con exactamente
              un v\'ertice en esa parte y por tanto, $G$ no es $k$-partita completa
              (no cumple el ser $k$-partita).
          \end{itemize}
        \item[$\Leftarrow$)] Procedamos reducción al absurdo .

          \begin{itemize}
            \item[$\cdot$)] Supongamos que $\overline{P_3} \subseteq G$, por
              definición de $k$-partita completa $\overline{P_3}$ no está en
              la misma parte (pues, en caso de estarlo hay una adyacencia en
              $2$ vértices de la misma parte). Luego $\overline{P_3}$ está en
              $2$ o $3$ partes distintas y habrá un $x \in \overline{P_3}$ que
              no se relacionará con al menos $1$ vértice en algunas de las partes
              y por tanto, $G$ no es $k$-partita completa!! (lo que no cumple es
              ser completa bajo el supuesto tomado) y he aquí una contradicción
              de suponer que $\overline{P_3} \subseteq G$. \\
              Por lo tanto, concluimos que $\overline{P_3} \nsubseteq G$.

            \item[$\cdot$)] Supongamos que $K_{k + 1} \subseteq G$, entonces hay $1$ vértice
              de $K_{k + 1}$ en cada una de las partes (lo que suma $k$ v\'ertices)
              y un $x \in K_{k + 1}$ en alguna parte tal que se relaciona con
              exactamente un v\'ertice en esa parte y por tanto, $G$ no es $k$-partita
              completa!! (no cumple el ser $k$-partita) y he aquí una contradicción
              de suponer $K_{k + 1} \subseteq G$. \\
              Por lo tanto, concluimos que $K_{k + 1} \nsubseteq G$.
          \end{itemize}
        \end{itemize}
        De los casos anterior concluimos que $G$ es $k$-partita completa si y
        s\'olo si no contiene a $K_{k+1}$ ni a $\overline{P_3}$ como subgr\'aficas
        inducidas.
      \end{proof}

      %%%%%%%%%%%%%%%%%%%%%%%%%%%%%%%% Ejercicio 2 %%%%%%%%%%%%%%%%%%%%%%%%%%%%%%%%
    \item Demuestre que si $G$ es una gr\'afica con $|V| \ge 4$ y $|E| > n^2/4$,
      entonces $G$ contiene un ciclo impar.

      %%%%%%%%%%%%%%%%%%%%%%%%%%%%%%%% Ejercicio 3 %%%%%%%%%%%%%%%%%%%%%%%%%%%%%%%%
    \item Sea $d = (d_1, \dots, d_n)$ una sucesi\'on no creciente de enteros no
      negativos. Sea $d' = (d_2-1, \dots, d_{d_1+1}-1, d_{d_1+2}, \dots, d_n)$.
      \begin{enumerate}
        \item Demuestre que $d$ es gr\'afica si y s\'olo si $d'$ es gr\'afica.

        \item Usando el primer inciso, describa un algoritmo que acepte como
          entrada una sucesi\'on no creciente de enteros no negativos $d$ y
          devuelva una gr\'afica simple con sucesi\'on de grados $d$, un
          certificado de que $d$ no es gr\'afica.
      \end{enumerate}
  \end{enumerate}
\end{document}
