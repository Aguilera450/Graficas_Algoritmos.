\documentclass{article}

% Symbols
%\usepackage{recycle}
\usepackage{amsfonts, amsthm}
\usepackage{upgreek}
\usepackage{physics}
\usepackage{cancel}
\usepackage{amssymb, latexsym, amsmath}

% Proof
\renewcommand*{\proofname}{\textbf{Demostraci\'on:}}

% Graphics
\usepackage{graphicx}
\usepackage{pgf}

% Color a letras.
%\usepackage[usenames,dvipsnames,svgnames,table]{xcolor}

% Tikz
\usepackage{tikz}
\usetikzlibrary{arrows,automata}
\usepackage{tikz}
\usetikzlibrary{arrows,automata}

\usetikzlibrary{shapes,calc}
\tikzstyle{edge}=[shorten <=2pt, shorten >=2pt,
  >=stealth, line width=1.1pt]
\tikzstyle{blueE}=[shorten <=2pt, shorten >=2pt,
  >=stealth, line width=1.5pt, blue]
\tikzstyle{blackV}=[circle, fill=black,
  minimum size=6pt,
  inner sep=0pt, outer sep=0pt]
\tikzstyle{blueV}=[circle, fill=blue, draw,
  minimum size=6pt, line width=0.75pt,
  inner sep=0pt, outer sep=0pt]
\tikzstyle{redV}=[circle, fill=red, draw,
  minimum size=6pt, line width=0.75pt,
  inner sep=0pt, outer sep=0pt]
\tikzstyle{redSV}=[semicircle, fill=red, minimum
  size=3pt, inner sep=0pt, outer sep=0pt,
  rotate=225]
\tikzstyle{blueSV}=[semicircle, fill=blue, minimum
  size=3pt, inner sep=0pt, outer sep=0pt,
  rotate=225]
\tikzstyle{blackSV}=[semicircle, fill=black, minimum
  size=3pt, inner sep=0pt, outer sep=0pt,
  rotate=225]
\tikzstyle{vertex}=[circle, draw, minimum size=6pt,
  line width=0.75pt, inner sep=0pt,
  outer sep=0pt]

% Margins
\addtolength{\voffset}{-0.5cm}
\addtolength{\hoffset}{-0.5cm}
\addtolength{\textwidth}{1cm}
\addtolength{\textheight}{1cm}

%Header-Footer
\usepackage{fancyhdr}
\renewcommand{\headrulewidth}{1pt}

\newcommand{\set}[1]{%
  \left\{ #1 \right\}%
}

%\pagenumbering{gobble} -- Este comando
%                       -- quita el número de página.
\footskip = 50pt
\renewcommand{\headrulewidth}{1pt}

\pagestyle{fancyplain}

\begin{document}
\title{UNIVERSIDAD AUT\'ONOMA DE M\'EXICO\\ Facultad de Ciencias}
\author{Autores:
  \\ Fernanda Villaf\'an Flores
  \\ Fernando Alvarado Palacios
  \\ Adri\'an Aguilera Moreno}
\date{}
\maketitle
\begin{center}
  \includegraphics[scale=0.20]{../Imagen/Portada.jpg}\\[0.4cm]
  \Large
  \bf{Gr\'aficas y Juegos}
  \normalsize
\end{center}
\newpage
\fancyhead[r]{ Gr\'aficas y Juegos 2022-1}
\section*{\LARGE{Tarea 2}}

\begin{enumerate}
  \small
  %%%%%%%%%%%%%%%%%%%%%%%%%%%%%%%% Ejercicio 1 %%%%%%%%%%%%%%%%%%%%%%%%%%%%%%%%
\item Demuestre que toda flecha en un camino dirigido cerrado en una digr\'afica
  pertenece a alg\'un ciclo dirigido.

  \begin{proof}
    Sea $C$ un camino dirigido cerrado tal que $C = (v_{0}, v_{1}, \dots, v_{i},
    \dots, v_{0})$.

    Procedamos por inducción sobre el número de veces que se repite un vértice
    distinto a $v_{0}$.

    \begin{itemize}
      \item \textbf{Paso Base:} $i = 0$.

        En este caso tenemos que el vértice $v_{i}$ no se repite ninguna vez, por
        lo que ya tenemos el ciclo dirigido buscado.

      \item \textbf{Hipótesis de Inducción:} Supongamos que $i > 0$.

      \item \textbf{Paso Inductivo:} Demostraremos que si el vértice $v_{i}$ se
      repite al menos una vez, hay un ciclo dirigido.

        Como el vértice $v_{i}$ se repite, podemos dividir a $C$ de la siguiente forma:

        Sea $C'$ un camino dirigido cerrado de $C$. \\
        Tenemos que:
        $$C' = (v_{i}, \dots, v_{i})$$
        Entonces, si $C'$ no repite vértices, ya tenemos el ciclo dirigido buscado. \\
        En caso contrario, aplicamos el mismo proceso:

        Sea $C''$ un camino dirigido cerrado de $C'$. \\
        Tenemos que:
        $$C'' = (v_{0}, \dots, v_{i}, v_{j}, \dots, v_{0}) \text{ , con $i < j$}$$

        Luego, este proceso lo realizamos hasta obtener un camino dirigido cerrado
        sin vértices repetidos.
    \end{itemize}
    Así, podemos concluir que toda flecha en un camino dirigido cerradoen una digráfica
    pertenece a algún ciclo dirigido.
  \end{proof}

  %%%%%%%%%%%%%%%%%%%%%%%%%%%%%%%% Ejercicio 2 %%%%%%%%%%%%%%%%%%%%%%%%%%%%%%%%
\item Demuestre que si $G$ es simple y $\delta \ge 2$, entonces $G$ contiene
  un ciclo de longitud al menos $\delta + 1$.

  \begin{proof}
    Sea $P$ una trayectoria de longitud máxima en $G$. \\
    Si $P = (v_{0}, \dots, v_{n})$, notemos que $v_{n}$ tiene un vecino distinto
    a $v_{n-1}$. \\
    Entonces, como $P$ es de longitud máxima, el otro vértice de $v_{n}$ debe ser
    un vértice en $P$ (digamos $v_{i}$ con $i < n-1$). \\
    Entonces $v_{i} P v_{n} v_{i}$ es un ciclo.
  \end{proof}
  %%%%%%%%%%%%%%%%%%%%%%%%%%%%%%%% Ejercicio 3 %%%%%%%%%%%%%%%%%%%%%%%%%%%%%%%%
\item Sea $G$ una gr\'afica conexa.   Demuestre que si $G$ no es completa,
  entonces contiente a $P_3$ como subgr\'afica inducida.
  \renewcommand\qedsymbol{QED}
  \begin{proof}
    Procedamos por reducción al absurdo.

    Sea $G$ una gráfica completa, entonces $P_{3}$ es subgráfica de $G$. Para este
    ejercicio necesitamos de una condición, $V_G \ge 3$, para las gráficas que no
    cumplan esto se tendrá la demostración por vacuidad.

    Tomemos a $x_{i - 1}, x_{i}, x_{i + 1}$ en $V_{G}$ ($2 \ge i \ge
    |V_{G}| - 1$), como $G$ es completa se tiene que la distancia entre cualesquiera
    $2$ v\'ertices es $1$, luego tenemos que hay ${3 \choose 2}$ aristas!! y esto es
    claramente mayor que $2$ ($|E_{P_3}|$), como los v\'ertices que tomamos son arbitrarios,
    podemos concluir que $P_{3} \nsubseteq G$.

    Como la anterior contradicci\'on resulta de suponer a $G$ completa, podemos
    asegurar que si $G$ no es completa, entonces $P_3 \subseteq G$.
  \end{proof}
  %%%%%%%%%%%%%%%%%%%%%%%%%%%%%%%% Ejercicio 4 %%%%%%%%%%%%%%%%%%%%%%%%%%%%%%%%
\item Demuestre que cualesquiera dos trayectorias de longitud m\'axima en una
  gr\'afica conexa tienen un vértice en común.
  %%%%%%%%%%%%%%%%%%%%%%%%%%%%%%%% Ejercicio 5 %%%%%%%%%%%%%%%%%%%%%%%%%%%%%%%%
\item Caracterice a las gr\'aficas $k$-regulares para $k \in \{ 0, 1, 2 \}$.

  \textit{\textbf{Soluci\'on:}}
  \begin{itemize}
  \item[$k = 0$)] Son todas las gráficas que no tienen aristas, a estas se les conoce
    como g\'aficas vac\'ias.
  \item[$k = 1$)] Estas son gr\'aficas con una cantidad de v\'ertices par y son tantas
    uniones de $P_2$ como $\frac{|V_G|}{2}$. Las gráficas con $|V_G|$ impar no entran
    aqu\'i porque siempre habr\'a $(|V_G| - 1) P_2$ y alg\'un v\'ertice (aislado) será
    de grado igual a $0$!! (o, pensando en lazos, de grado igual a 2), lo que contradice
    el ser $1-$regular.
  \item[$k = 2$)] Son gr\'aficas que contienen ciclos o son combinaciones de ciclos.
    Todos los ciclos son $2-$regulares, esto no implica que todas las gr\'aficas
    $2-$regulares sean un ciclo pero si que los contengan o que sean combinaciones
    de estos.
  \end{itemize}
  Con los $3$ puntos anteriores concluimos la caracterización. \hfill $\square$
  %%%%%%%%%%%%%%%%%%%%%%%%%%%%%%%% Ejercicio 6 %%%%%%%%%%%%%%%%%%%%%%%%%%%%%%%%
\item Demuestre que si $|E| \ge |V|$, entonces $G$ contiene un ciclo.

\begin{proof} (Por contradicción)

  Supongamos que G es una gráfica tal que $|E|\geq|V|$ y G no contiene ningun ciclo $\Longrightarrow$ el número máximo de vétices de este tipo de gráficas será igual a $|V|-1$ (Ya que todo vértice se podria relacionar con algun otro vertice sin repetir con los vértices anteriores, pero el último no se podra relacionar con algun otro vertice ya que si lo hiciera formaría un ciclo) $\Longrightarrow$ $|E|\leq|V|-1 < |V|$ (!lo que es una contradiccón ya que contradice nuestra hipótesis de que $|E|\geq|V|$). Por lo tanto la gráfica G debe tener al menos un ciclo.

  \end{proof}
\end{enumerate}

\section*{Puntos extra}

\begin{enumerate}
  %%%%%%%%%%%%%%%%%%%%%%%%%%%%%%%% Ejercicio 1 %%%%%%%%%%%%%%%%%%%%%%%%%%%%%%%%
\item Sea $G$ una gr\'afica.   Demuestre que $G$ es $k$-partita completa si y
  s\'olo si no contiene a $K_{k+1}$ ni a $\overline{P_3}$ como subgr\'aficas
  inducidas.
  
  \begin{proof}
    En este ejercicio analizaremos 2 casos posibles:
    
    \begin{itemize}
    \item[$\Rightarrow$)] Procedamos por contrapositiva.

      \begin{itemize}
      \item[$\cdot$)] Si $\overline{P_3} \subseteq G$, por definición
        de $k$-partita completa $\overline{P_3}$ no esta en la misma
        parte (pues, en caso de estarlo hay una adyacencia en $2$ vértices
        de la misma parte), luego $\overline{P_3}$ esta en $2$ o $3$
        partes distintas y habrá un $x \in \overline{P_3}$ que no se
        relacionará con al menos $1$ vértice en algunas de las partes
        y por tanto $G$ no es $k$-partita completa (lo que no cumple es
        ser completa bajo el supuesto tomado).
        
      \item[$\cdot$)] Si $K_{k + 1} \subseteq G$, entonces hay $1$ vértice
        de $K_{k + 1}$ en cada una de las partes (lo que suma $k$ v\'ertices) y un
        $x \in K_{k + 1}$ en alguna parte tal que se relaciona con exactamente
        un v\'ertice en esa parte y por tanto $G$ no es $k$-partita completa
        (no cumple el ser $k$-partita).
      \end{itemize}
    \item[$\Leftarrow$)] Procedamos reducción al absurdo .
      
      \begin{itemize}
      \item[$\cdot$)] Supongamos que $\overline{P_3} \subseteq G$, por
        definición de $k$-partita completa $\overline{P_3}$ no esta en
        la misma parte (pues, en caso de estarlo hay una adyacencia en
        $2$ vértices de la misma parte), luego $\overline{P_3}$ esta en
        $2$ o $3$ partes distintas y habrá un $x \in \overline{P_3}$ que
        no se relacionará con al menos $1$ vértice en algunas de las partes
        y por tanto $G$ no es $k$-partita completa !!(lo que no cumple es
        ser completa bajo el supuesto tomado) y he aquí una contradicción
        de suponer que $\overline{P_3} \subseteq G$. Por tanto concluimos
        que $\overline{P_3} \nsubseteq G$.
        
      \item[$\cdot$)] Supongamos que $K_{k + 1} \subseteq G$, entonces hay $1$ vértice
        de $K_{k + 1}$ en cada una de las partes (lo que suma $k$ v\'ertices)
        y un $x \in K_{k + 1}$ en alguna parte tal que se relaciona con
        exactamente un v\'ertice en esa parte y por tanto $G$ no es $k$-partita
        completa!! (no cumple el ser $k$-partita) y he aquí una contradicción
        de suponer $K_{k + 1} \subseteq G$. Por tanto concluimos que
        $K_{k + 1} \nsubseteq G$
      \end{itemize}
    \end{itemize}
    De los casos anterior concluimos que $G$ es $k$-partita completa si y
  s\'olo si no contiene a $K_{k+1}$ ni a $\overline{P_3}$ como subgr\'aficas
  inducidas.
  \end{proof}
  %%%%%%%%%%%%%%%%%%%%%%%%%%%%%%%% Ejercicio 2 %%%%%%%%%%%%%%%%%%%%%%%%%%%%%%%%
\item Demuestre que si $G$ es una gr\'afica con $|V| \ge 4$ y $|E| > n^2/4$,
  entonces $G$ contiene un ciclo impar.
  %%%%%%%%%%%%%%%%%%%%%%%%%%%%%%%% Ejercicio 3 %%%%%%%%%%%%%%%%%%%%%%%%%%%%%%%%
\item Sea $d = (d_1, \dots, d_n)$ una sucesi\'on no creciente de enteros no
  negativos. Sea $d' = (d_2-1, \dots, d_{d_1+1}-1, d_{d_1+2}, \dots, d_n)$.
  \begin{enumerate}
  \item Demuestre que $d$ es gr\'afica si y s\'olo si $d'$ es gr\'afica.

  \item Usando el primer inciso, describa un algoritmo que acepte como
    entrada una sucesi\'on no creciente de enteros no negativos $d$ y
    devuelva una gr\'afica simple con sucesi\'on de grados $d$, un
    certificado de que $d$ no es gr\'afica.
  \end{enumerate}
\end{enumerate}

\end{document}
