\documentclass{article}

% Symbols
%\usepackage{recycle}
\usepackage{amsfonts, amsthm}
\usepackage{upgreek}
\usepackage{physics}
\usepackage{cancel}
\usepackage{amssymb, latexsym, amsmath}

% Proof
\renewcommand*{\proofname}{\textbf{Demostraci\'on:}}

% Graphics
\usepackage{graphicx}
\usepackage{pgf}

% Color a letras.
%\usepackage[usenames,dvipsnames,svgnames,table]{xcolor}

% Tikz
\usepackage{tikz}
\usetikzlibrary{arrows,automata}
\usepackage{tikz}
\usetikzlibrary{arrows,automata}

\usetikzlibrary{shapes,calc}
\tikzstyle{edge}=[shorten <=2pt, shorten >=2pt,
  >=stealth, line width=1.1pt]
\tikzstyle{blueE}=[shorten <=2pt, shorten >=2pt,
  >=stealth, line width=1.5pt, blue]
\tikzstyle{blackV}=[circle, fill=black,
  minimum size=6pt,
  inner sep=0pt, outer sep=0pt]
\tikzstyle{blueV}=[circle, fill=blue, draw,
  minimum size=6pt, line width=0.75pt,
  inner sep=0pt, outer sep=0pt]
\tikzstyle{redV}=[circle, fill=red, draw,
  minimum size=6pt, line width=0.75pt,
  inner sep=0pt, outer sep=0pt]
\tikzstyle{redSV}=[semicircle, fill=red, minimum
  size=3pt, inner sep=0pt, outer sep=0pt,
  rotate=225]
\tikzstyle{blueSV}=[semicircle, fill=blue, minimum
  size=3pt, inner sep=0pt, outer sep=0pt,
  rotate=225]
\tikzstyle{blackSV}=[semicircle, fill=black, minimum
  size=3pt, inner sep=0pt, outer sep=0pt,
  rotate=225]
\tikzstyle{vertex}=[circle, draw, minimum size=6pt,
  line width=0.75pt, inner sep=0pt,
  outer sep=0pt]

% Margins
\addtolength{\voffset}{-0.5cm}
\addtolength{\hoffset}{-0.5cm}
\addtolength{\textwidth}{1cm}
\addtolength{\textheight}{1cm}

%Header-Footer
\usepackage{fancyhdr}
\renewcommand{\headrulewidth}{1pt}

\newcommand{\set}[1]{%
  \left\{ #1 \right\}%
}

%\pagenumbering{gobble} -- Este comando
%                       -- quita el número de página.
\footskip = 50pt
\renewcommand{\headrulewidth}{1pt}

\pagestyle{fancyplain}

\begin{document}
\title{UNIVERSIDAD AUT\'ONOMA DE M\'EXICO\\ Facultad de Ciencias}
\author{Autores:
  \\ Fernanda Villaf\'an Flores
  \\ Fernando Alvarado Palacios
  \\ Adri\'an Aguilera Moreno}
\date{}
\maketitle
\begin{center}
  \includegraphics[scale=0.20]{../Imagen/Portada.jpg}\\[0.4cm]
  \Large
  \bf{Gr\'aficas y Juegos}
  \normalsize
\end{center}
\newpage
\fancyhead[r]{ Gr\'aficas y Juegos 2022-1}
\section*{\LARGE{Tarea 2}}

\begin{enumerate}
  \small
  %%%%%%%%%%%%%%%%%%%%%%%%%%%%%%%% Ejercicio 1 %%%%%%%%%%%%%%%%%%%%%%%%%%%%%%%%
\item Demuestre que toda flecha en un camino dirigido cerrado en una  digr\'afica
  pertenece a alg\'un ciclo dirigido.
  %%%%%%%%%%%%%%%%%%%%%%%%%%%%%%%% Ejercicio 2 %%%%%%%%%%%%%%%%%%%%%%%%%%%%%%%%
\item Demuestre que si $G$ es simple y $\delta \ge 2$, entonces $G$ contiene
  un ciclo de longitud al menos $\delta + 1$.
  %%%%%%%%%%%%%%%%%%%%%%%%%%%%%%%% Ejercicio 3 %%%%%%%%%%%%%%%%%%%%%%%%%%%%%%%%
\item Sea $G$ una gr\'afica conexa.   Demuestre que si $G$ no es completa,
  entonces contiente a $P_3$ como subgr\'afica inducida.
  \renewcommand\qedsymbol{QED}
  \begin{proof}
    Procedamos por reducción al absurdo.

    Sea $G$ una gráfica completa, entonces $P_{3}$ es subgráfica de $G$. Para este
    ejercicio necesitamos de una condición, $V_G \ge 3$, para las gráficas que no
    cumplan esto se tendrá la demostración por vacuidad.

    Tomemos a $x_{i - 1}, x_{i}, x_{i + 1}$ en $V_{G}$ ($2 \ge i \ge
    |V_{G}| - 1$), como $G$ es completa se tiene que la distancia entre cualesquiera
    $2$ v\'ertices es $1$, luego tenemos que hay ${3 \choose 2}$ aristas!! y esto es
    claramente mayor que $2$ ($|E_{P_3}|$), como los v\'ertices que tomamos son arbitrarios,
    podemos concluir que $P_{3} \nsubseteq G$.
    
    Como la anterior contradicci\'on resulta de suponer a $G$ completa, podemos
    asegurar que si $G$ no es completa, entonces $P_3 \subseteq G$.
  \end{proof}
  %%%%%%%%%%%%%%%%%%%%%%%%%%%%%%%% Ejercicio 4 %%%%%%%%%%%%%%%%%%%%%%%%%%%%%%%%
\item Demuestre que cualesquiera dos trayectorias de longitud m\'axima en una
  gr\'afica conexa tienen un vértice en común.
  %%%%%%%%%%%%%%%%%%%%%%%%%%%%%%%% Ejercicio 5 %%%%%%%%%%%%%%%%%%%%%%%%%%%%%%%%
\item Caracterice a las gr\'aficas $k$-regulares para $k \in \{ 0, 1, 2 \}$.
  
  \textit{\textbf{Soluci\'on:}}
  \begin{itemize}
  \item[$k = 0$)] Son todas las gráficas que no tienen aristas, a estas se les conoce
    como g\'aficas vac\'ias.
  \item[$k = 1$)] Estas son gr\'aficas con una cantidad de v\'ertices par y son tantas
    uniones de $P_2$ como $\frac{|V_G|}{2}$. Las gráficas con $|V_G|$ impar no entran
    aqu\'i porque siempre habr\'a $(|V_G| - 1) P_2$ y alg\'un v\'ertice (aislado) será
    de grado igual a $0$!! (o, pensando en lazos, de grado igual a 2), lo que contradice
    el ser $1-$regular.
  \item[$k = 2$)] Son gr\'aficas que contienen ciclos o son combinaciones de ciclos.
    Todos los ciclos son $2-$regulares, esto no implica que todas las gr\'aficas
    $2-$regulares sean un ciclo pero si que los contengan o que sean combinaciones
    de estos.
  \end{itemize}
  Con los $3$ puntos anteriores concluimos la caracterización. \hfill $\square$
  %%%%%%%%%%%%%%%%%%%%%%%%%%%%%%%% Ejercicio 6 %%%%%%%%%%%%%%%%%%%%%%%%%%%%%%%%
\item Demuestre que si $|E| \ge |V|$, entonces $G$ contiene un ciclo.

\begin{proof} (Por contradicción)

  Supongamos que G es una gráfica tal que $|E|\geq|V|$ y G no contiene ningun ciclo $\Longrightarrow$ el número máximo de vétices de este tipo de gráficas será igual a $|V|-1$ (Ya que todo vértice se podria relacionar con algun otro vertice sin repetir con los vértices anteriores, pero el último no se podra relacionar con algun otro vertice ya que si lo hiciera formaría un ciclo) $\Longrightarrow$ $|E|\leq|V|-1 < |V|$ (!lo que es una contradiccón ya que contradice nuestra hipótesis de que $|E|\geq|V|$). Por lo tanto la gráfica G debe tener al menos un ciclo.
  
  \end{proof}
\end{enumerate}

\section*{Puntos extra}

\begin{enumerate}
  %%%%%%%%%%%%%%%%%%%%%%%%%%%%%%%% Ejercicio 1 %%%%%%%%%%%%%%%%%%%%%%%%%%%%%%%%
\item Sea $G$ una gr\'afica.   Demuestre que $G$ es $k$-partita completa si y
  s\'olo si no contiene a $K_{k+1}$ ni a $\overline{P_3}$ como subgr\'aficas
  inducidas.
  %%%%%%%%%%%%%%%%%%%%%%%%%%%%%%%% Ejercicio 2 %%%%%%%%%%%%%%%%%%%%%%%%%%%%%%%%
\item Demuestre que si $G$ es una gr\'afica con $|V| \ge 4$ y $|E| > n^2/4$,
  entonces $G$ contiene un ciclo impar.
  %%%%%%%%%%%%%%%%%%%%%%%%%%%%%%%% Ejercicio 3 %%%%%%%%%%%%%%%%%%%%%%%%%%%%%%%%
\item Sea $d = (d_1, \dots, d_n)$ una sucesi\'on no creciente de enteros no
  negativos. Sea $d' = (d_2-1, \dots, d_{d_1+1}-1, d_{d_1+2}, \dots, d_n)$.
  \begin{enumerate}
  \item Demuestre que $d$ es gr\'afica si y s\'olo si $d'$ es gr\'afica.

  \item Usando el primer inciso, describa un algoritmo que acepte como
    entrada una sucesi\'on no creciente de enteros no negativos $d$ y
    devuelva una gr\'afica simple con sucesi\'on de grados $d$, un
    certificado de que $d$ no es gr\'afica.
  \end{enumerate}
\end{enumerate}

\end{document}
