\documentclass{article}

% Symbols
%\usepackage{recycle}
\usepackage{amsfonts, amsthm}
\usepackage{upgreek}
\usepackage{physics}
\usepackage{cancel}
\usepackage{amssymb, latexsym, amsmath}

% Proof
\renewcommand*{\proofname}{\textbf{Demostraci\'on:}}

% Graphics
\usepackage{graphicx}
\usepackage{pgf}

% Color a letras.
%\usepackage[usenames,dvipsnames,svgnames,table]{xcolor}

% Tikz
\usepackage{tkz-graph}
\usepackage{tikz}
\usetikzlibrary{arrows,automata}
\usepackage{tikz}
\usetikzlibrary{arrows,automata}
% Def. Dr. César.
\usetikzlibrary{shapes,calc}
\tikzstyle{edge}=[shorten <=2pt, shorten >=2pt, >=stealth, line width=1.1pt]
\tikzstyle{blueE}=[shorten <=2pt, shorten >=2pt, >=stealth, line width=1.5pt, blue]
\tikzstyle{blackV}=[circle, fill=black, minimum size=6pt, inner sep=0pt, outer sep=0pt]
\tikzstyle{blueV}=[circle, fill=blue, draw, minimum size=6pt, line width=0.75pt, inner sep=0pt, outer sep=0pt]
\tikzstyle{redV}=[circle, fill=red, draw, minimum size=6pt, line width=0.75pt, inner sep=0pt, outer sep=0pt]
\tikzstyle{redSV}=[semicircle, fill=red, minimum size=3pt, inner sep=0pt, outer sep=0pt, rotate=225]
\tikzstyle{blueSV}=[semicircle, fill=blue, minimum size=3pt, inner sep=0pt, outer sep=0pt, rotate=225]
\tikzstyle{blackSV}=[semicircle, fill=black, minimum size=3pt, inner sep=0pt, outer sep=0pt, rotate=225]
\tikzstyle{vertex}=[circle, draw, minimum size=6pt, line width=0.75pt, inner sep=0pt, outer sep=0pt]

% Margins
\addtolength{\voffset}{-1cm}
\addtolength{\hoffset}{-1cm}
\addtolength{\textwidth}{2cm}
\addtolength{\textheight}{2cm}

%Header-Footer
\usepackage{fancyhdr}
\renewcommand{\headrulewidth}{1pt}

\newcommand{\set}[1]{%
  \left\{ #1 \right\}%
}

%\pagenumbering{gobble} -- Este comando
%                       -- quita el número de página.
\footskip = 50pt
\renewcommand{\headrulewidth}{1pt}

\pagestyle{fancyplain}

\begin{document}
\title{UNIVERSIDAD AUT\'ONOMA DE M\'EXICO\\ Facultad de Ciencias}
\author{Autores:
  \\ Fernanda Villaf\'an Flores
  \\ Fernando Alvarado Palacios
  \\ Adri\'an Aguilera Moreno}
\date{}
\maketitle
\begin{center}
  \includegraphics[scale=0.20]{../Imagen/Portada.jpg}\\[0.4cm]
  \Large
  \bf{Gr\'aficas y Juegos}
  \normalsize
\end{center}
\newpage
\fancyhead[r]{ Gr\'aficas y Juegos 2022-1}
%%%%%%%%%%%%%%%%%%%%%%%%%%%%%%%%%%%%%%%%%%%%%%%%%%%%%
\section*{\LARGE{Tarea 5}}
\begin{enumerate}
\item Demuestre que si $G$ es simple y $3$-regular, entonces $\kappa =
  \kappa'$.

\item Demuestre que una gr\'afica es $2$-conexa por aristas si y s\'olo si
  cualesquiera dos v\'ertices est\'an conectados por al menos dos trayectorias
  ajenas por aristas.

\item Demuestre que si $G$ no tiene ciclos pares, entonces cada bloque de $G$
  es $K_1$, $K_2$ o un ciclo impar.

\item Sea $G$ una gr\'afica $2$-conexa y sean $X$ y $Y$ subconjuntos ajenos de
  $V$, cada un con al menos dos v\'ertices.   Demuestre que $G$ contiene
  trayectorias ajenas $P$ y $Q$ tales que
  \begin{enumerate}
  \item Los v\'ertices iniciales de $P$ y $Q$ pertenecen a $X$.

  \item Los v\'ertices finales de $P$ y $Q$ pertenecen a $Y$.

  \item Ning\'un v\'ertice interno de $P$ o $Q$ pertenece a $X \cup Y$.
  \end{enumerate}

\item Sea $G$ una gr\'afica conexa con al menos 3 v\'ertices. Demuestre que
  los siguientes enunciados son equivalentes.
  \begin{enumerate}
  \item $G$ es un bloque.

  \item Entre cualesquiera dos v\'ertices distintos existen dos trayectorias
    internamente ajenas.

  \item Para cualesquiera dos v\'ertices de $G$ existe un ciclo que los
    contiene.

  \item Para cualquier v\'ertice y cualquier arista de $G$ existe un ciclo
    que los contiene.

  \item Para cualesquiera dos aristas de $G$ existe un ciclo que los
    contiene.

  \item Dados dos v\'ertices $u,v \in V(G)$ y una arista $e \in E(G)$,
    existe una $uv$-trayectoria que pasa por $e$.

  \item Para cualesquiera tres v\'ertices distintos de $G$, existe una
    trayectoria que une a cualesquiera dos de ellos y que pasa por el
    tercero.

  \item Para cualesquiera tres v\'ertices distintos de $G$, existe una
    trayectoria que une a cualesquiera dos de ellos que no pasa por el
    tercero.
  \end{enumerate}
\end{enumerate}

\end{document}
