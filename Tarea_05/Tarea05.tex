\documentclass{article}

% Symbols
%\usepackage{recycle}
\usepackage{amsfonts, amsthm}
\usepackage{upgreek}
\usepackage{physics}
\usepackage{cancel}
\usepackage{amssymb, latexsym, amsmath}

% Proof
\renewcommand*{\proofname}{\textbf{Demostraci\'on:}}

% Graphics
\usepackage{graphicx}
\usepackage{pgf}

% Color a letras.
%\usepackage[usenames,dvipsnames,svgnames,table]{xcolor}

% Tikz
\usepackage{tkz-graph}
\usepackage{tikz}
\usetikzlibrary{arrows,automata}
\usepackage{tikz}
\usetikzlibrary{arrows,automata}
% Def. Dr. César.
\usetikzlibrary{shapes,calc}
\tikzstyle{edge}=[shorten <=2pt, shorten >=2pt, >=stealth, line width=1.1pt]
\tikzstyle{blueE}=[shorten <=2pt, shorten >=2pt, >=stealth, line width=1.5pt, blue]
\tikzstyle{blackV}=[circle, fill=black, minimum size=6pt, inner sep=0pt, outer sep=0pt]
\tikzstyle{blueV}=[circle, fill=blue, draw, minimum size=6pt, line width=0.75pt, inner sep=0pt, outer sep=0pt]
\tikzstyle{redV}=[circle, fill=red, draw, minimum size=6pt, line width=0.75pt, inner sep=0pt, outer sep=0pt]
\tikzstyle{redSV}=[semicircle, fill=red, minimum size=3pt, inner sep=0pt, outer sep=0pt, rotate=225]
\tikzstyle{blueSV}=[semicircle, fill=blue, minimum size=3pt, inner sep=0pt, outer sep=0pt, rotate=225]
\tikzstyle{blackSV}=[semicircle, fill=black, minimum size=3pt, inner sep=0pt, outer sep=0pt, rotate=225]
\tikzstyle{vertex}=[circle, draw, minimum size=6pt, line width=0.75pt, inner sep=0pt, outer sep=0pt]

% Margins
\addtolength{\voffset}{-1cm}
\addtolength{\hoffset}{-1cm}
\addtolength{\textwidth}{2cm}
\addtolength{\textheight}{2cm}

%Header-Footer
\usepackage{fancyhdr}
\renewcommand{\headrulewidth}{1pt}

\newcommand{\set}[1]{%
  \left\{ #1 \right\}%
}

%\pagenumbering{gobble} -- Este comando
%                       -- quita el número de página.
\footskip = 50pt
\renewcommand{\headrulewidth}{1pt}

\pagestyle{fancyplain}

\begin{document}
\title{UNIVERSIDAD AUT\'ONOMA DE M\'EXICO\\ Facultad de Ciencias}
\author{Autores:
  \\ Fernanda Villaf\'an Flores
  \\ Fernando Alvarado Palacios
  \\ Adri\'an Aguilera Moreno}
\date{}
\maketitle
\begin{center}
  \includegraphics[scale=0.20]{../Imagen/Portada.jpg}\\[0.4cm]
  \Large
  \bf{Gr\'aficas y Juegos}
  \normalsize
\end{center}
\newpage
\fancyhead[r]{ Gr\'aficas y Juegos 2022-1}
%%%%%%%%%%%%%%%%%%%%%%%%%%%%%%%%%%%%%%%%%%%%%%%%%%%%%
\section*{\LARGE{Tarea 5}}
\begin{enumerate}
  %%%%%%%%%%%%%%%%%%%%%%%%%%%%%%%%% Ejercicio 01 %%%%%%%%%%%%%%%%%%%%%%%%%%%%%%%%%%%%%
\item Demuestre que si $G$ es simple y $3$-regular, entonces $\kappa =
  \kappa'$.
  %%%%%%%%%%%%%%%%%%%%%%%%%%%%%%%%% Ejercicio 02 %%%%%%%%%%%%%%%%%%%%%%%%%%%%%%%%%%%%%
\item Demuestre que una gr\'afica es $2$-conexa por aristas si y s\'olo si
  cualesquiera dos v\'ertices est\'an conectados por al menos dos trayectorias
  ajenas por aristas.
  %%%%%%%%%%%%%%%%%%%%%%%%%%%%%%%%% Ejercicio 03 %%%%%%%%%%%%%%%%%%%%%%%%%%%%%%%%%%%%%
\item Demuestre que si $G$ no tiene ciclos pares, entonces cada bloque de $G$
  es $K_1$, $K_2$ o un ciclo impar.
  %%%%%%%%%%%%%%%%%%%%%%%%%%%%%%%%% Ejercicio 04 %%%%%%%%%%%%%%%%%%%%%%%%%%%%%%%%%%%%%
\item Sea $G$ una gr\'afica $2$-conexa y sean $X$ y $Y$ subconjuntos ajenos de
  $V$, cada un con al menos dos v\'ertices.   Demuestre que $G$ contiene
  trayectorias ajenas $P$ y $Q$ tales que
  \begin{enumerate}
  \item Los v\'ertices iniciales de $P$ y $Q$ pertenecen a $X$.

  \item Los v\'ertices finales de $P$ y $Q$ pertenecen a $Y$.

  \item Ning\'un v\'ertice interno de $P$ o $Q$ pertenece a $X \cup Y$.
  \end{enumerate} 

  \begin{proof} 

    Sea G 2-conexa y X, Y subconjuntos ajenos de V $\Leftrightarrow$ sean x'$\in$ X y y'$\in$ Y $\Leftrightarrow$ por el ejercicio 2 esta tarea, podemos asegurar que existen dos trayectorias internamente ajenas por aristas que conectan a x' y y', nombremos a estas trayectorias P' y Q'.
    
    Sea P'$=(x'=a_0, a_1, a_2, ..., a_n = y')$ para alguna n que pertenezca a los Naturales y sea  Q'$=(x' =b_0, b_1, b_2, ..., b_k = y')$ para alguna k que pertenezca a los naturales.
    
    $\Leftrightarrow$ Comparando los vertices, sea $a_i$ para alguna i que pertenece a los naturales, tal que  $a_i \in$ X, pero  $a_{i+1} \notin$ X por lo que ahora P'$=(x'=a_i, a_{i+1}, a_{i+2}, ..., a_n = y')$.
    
    De igual forma existira algun $b_r$ para alguna r que pertenece a los naturales, tal que $b_r \in$ X, pero $b_{r+1} \notin $ X por lo que ahora  Q'$=(x' =b_r, b_{r+1}, b_{r+2}, ..., b_k = y')$.
    
    Ahora definamos $P'^{-1}$ como $(y'=a_n, a_{n-1}, a_{n-2}, ..., a_i=x')$ y $Q'^{-1}$ como $(y'=b_k, b_{k-1}, b_{k-2}, ..., b_r=x') \Leftrightarrow$ aplicando el razonamiento anterior,  existira t que pertenenzca los naturales, tal que $b_{k-t} \in Y$ y $b_{k-(t+1)} \notin Y$  y de igual forma existiara s que pertenezca a los naturales tal que $a_{n-s} \in Y$ y  $a_{n-(s+1)} \notin Y \Leftrightarrow$ definamos a $P=(a_i, a_{i+1}, ...., a_{n-s})$ y definimos a $Q=(b_j, b_{j+1}, ..., b_{k-t})$
    
    Por lo tanto: 
    
    a) se cumple ya que $a_i$ y $b_j$ pertenencen a X por construccion de nuetras trayectorias P y Q.
    
    b) se cumple ya que $a_{n-r}$ y $b_{k-t}$ pertenencen a Y por construccion de P y Q.
    
    c) se cumple de nuevo por construccion de P y Q y ya que X y Y son conjuntos disjuntos de V
    
    \end{proof}

  %%%%%%%%%%%%%%%%%%%%%%%%%%%%%%%%% Ejercicio 05 %%%%%%%%%%%%%%%%%%%%%%%%%%%%%%%%%%%%%
\item Sea $G$ una gr\'afica conexa con al menos 3 v\'ertices. Demuestre que
  los siguientes enunciados son equivalentes.
  \begin{enumerate}
  \item $G$ es un bloque.

  \item Entre cualesquiera dos v\'ertices distintos existen dos trayectorias
    internamente ajenas.

  \item Para cualesquiera dos v\'ertices de $G$ existe un ciclo que los
    contiene.

  \item Para cualquier v\'ertice y cualquier arista de $G$ existe un ciclo
    que los contiene.

  \item Para cualesquiera dos aristas de $G$ existe un ciclo que los
    contiene.

  \item Dados dos v\'ertices $u,v \in V(G)$ y una arista $e \in E(G)$,
    existe una $uv$-trayectoria que pasa por $e$.

  \item Para cualesquiera tres v\'ertices distintos de $G$, existe una
    trayectoria que une a cualesquiera dos de ellos y que pasa por el
    tercero.

  \item Para cualesquiera tres v\'ertices distintos de $G$, existe una
    trayectoria que une a cualesquiera dos de ellos que no pasa por el
    tercero.
  \end{enumerate}
  
  \renewcommand\qedsymbol{QED}
  \begin{proof}  
    Las implicaciones; $(a) \Rightarrow (b)$, $(b) \Rightarrow (c)$
    y $(c) \Rightarrow (d)$, se omiten por estar escritas en las
    notas de clase. Las implicaciones restantes se enumeran a
    continuación:
    \begin{itemize}
    \item[$\cdot$)] Por mostrar $(d) \Rightarrow (e)$.
      
      Por ($d$) sabemos que cualquier arista se encuentra
      contenida en un ciclo, por consecuencia directa,
      cualesquiera $2$ aristas se encuentran contenidas en
      algún ciclo.
    \item[$\cdot$)] Por mostrar $(e) \Rightarrow (f)$.
      
      Nótese que como consecuencia de ($e$)(en partícular
      cualquier arista esta contenida en un ciclo)
      cualquier vértice no aislado forma parte de un ciclo,
      como $G$ es conexa, entonces todos sus vértices son
      parte de algún ciclo, esto nos lleva a que $G$ es
      $2$-conexa por aristas. Luego para $u,v$ en $V_G$
      existen al menos dos $uv$-trayectoria ajenas por
      aristas. De lo anterior veamos como son $v$ y $u$,
      esto es
      \begin{itemize}
      \item Si $u, v$ forman parte de un mismo ciclo $C$.
        Sea $e \in E_G$ tal que forme parte de $C$, entonces
        terminamos, pues existe una trayectoria que pasa por
        $e$. Ahora, supongamos que $e$ no es parte de $C$, y
        sea $e = xy (x, y \text{ en } V_G)$, así supongamos que
        los vértices $y$ y $v$ son los que se pueden "conectar"
        por la trayectoria de mayor longitud entre ellos y a la
        que llamaremos $T$, veamos que si $x$ se encuentra en $T$
        terminamos, pues existe una $xu$-trayectoria que complementa
        a $T$ para formar una $uv$-trayectoria con $e$ contenida, luego
        si $x$ no esta contenida en $T$, entonces $Tx$ incluye a $e$ por
        lo que si $u$ no esta contenida en $T$, entonces existe una
        $xu$-trayectoria $P$ tal que $TxP$ es la trayectoria buscada\footnote{
          Con esto debería bastar, pues estamos mostrando la equivalencia
          a un bloque propiamente, sin embargo el caso restante también
          lo analizo, pues no veo como descartalo sin usar lo que estoy
          mostrando.}.
        
      \item $u, v$ forman parte de ciclos distintos. Si $e$ forma
        parte de alguno de los ciclos que contiene a $u$ y $v$ como
        vértices, entonces hay una trayectoria $Q$ que pasa por $v$
        (o $u$) y por $e$, pues están en el mismo ciclo, luego existe
        una trayectoria $P$ que inicia en $e$ y llega hasta $u$ (o $v$)
        como consecuencia de que $G$ sea conexa, además $P$ ajena con $Q$
        por ser $G$ $2$-conexa por aristas, luego $PQ$ es la trayectoria
        que pasa por $u$ y $v$ que además contiene a la arista $e$. Para
        finalizar, si $e$ no se encuentra contenida en los ciclos que
        contienen a $u$ y $v$ como vértices. Supongamos que
        $e = xy (x, y \text{ en } V_G)$ y que de $x$ se pueda obtener la
        trayectoria más larga con $v$, así, existe una $xv$-trayectoria $T$
        que ya contiene a $e$ (porque $T$ es la más larga, caso contrario
        sólo basta unir $T = Ty$), como existe una $vu$-trayectoria $P$,
        entonces $TP$ ($TyP$) es la trayectoria buscada y terminamos.
      \end{itemize}
    \item[$\cdot$)] Por mostrar $(f) \Rightarrow (g)$.
      
      Por ($f$) tenemos que cualesquiera $2$ vértices $u, v$ y para
      toda arista $e$, existe una $uv$-trayectoria que contiene a $e$.
      Como $G$ es conexa, entonces, cualquier vértice es parte de alguna
      arista, así, para cualesquiera $x, y, z$ en $V_G$ existe una
      $xz$-trayectoria $T$ que contiene a $e \in E_G$, tal que $e =
      yw (w \in E_G)$ y por tanto $T$ pasa por $y$.
    \item[$\cdot$)] Por mostrar $(g) \Rightarrow (h)$.
      
      Como $G$ es conexa, sabemos que existe una $xy$-trayectoria $T$,
      una $xz$-trayectoria $P$, y una $yz$-trayectoria $Q$, todas ellas
      de longitud mínima. Así
      \begin{itemize}
      \item[-] Si todas las longitudes son iguales, \textit{i.e.}, $\mathcal{L}(T)
        = \mathcal{L}(P) = \mathcal{L}(Q)$. Por hipótesis $x,y,z$ no son iguales,
        así, cualquier trayectoria no contiene al tercer vértice y hemos encontrado
        al menos $2$ trayectoias que cumplen con $h$.
        
      \item[-] Si dos de las trayectorias tienen una longitus mínima igual y menor
        a la que es desigual, \textit{i.e.}, $\mathcal{L}(T) = \mathcal{L}(P) < \mathcal{L}(Q)$
        o $\mathcal{L}(T) > \mathcal{L}(P) = \mathcal{L}(Q)$ o $\mathcal{L}(T) = \mathcal{L}(Q)
        < \mathcal{L}(P)$. En este caso, analicemos las dos trayectorias que resulten tener menor
        longitud entre $P, Q, T$. Si las trayectorias son $T$ y $P$, entonces cualquiera de estas
        no contiene al tercer vértice, pues $x,y,z$ son distintos y las trayectorias elegidas son
        las de menor longitud. Si se tiene que las trayectorias mínimas son $T$ y $Q$, o $P$ y $Q$,
        se cumple que un tercer vértice no esta en alguna de las dos trayectorias, pues en caso
        contrario y por saber que $x,y,z$ son distintos, se tendría que  hay una trayectoria menor
        con $2$ de los vértices que estamos trabajando!!!, pero esto es absurdo, ya que, nuestras
        trayectorias eran mínimas, por lo cual se cumple $h$.
        
      \item[-] Si una de las longitudes mínimas es menor a las restantes, \textit{i.e.},
        $\mathcal{L}(T) = \mathcal{L}(P) > \mathcal{L}(Q)$ o $\mathcal{L}(T) < \mathcal{L}(P)
        = \mathcal{L}(Q)$ o $\mathcal{L}(T) = \mathcal{L}(Q) > \mathcal{L}(P)$. Para este caso,
        elegimos la trayectoria con menor longitud y esta no contendrá al tercer vértice, pues
        para que pasará por el se necesitaría recorrer más aristas. En este caso tenemos una
        trayectoria que cumple con $h$.
      \end{itemize}
      
    \item[$\cdot$)] Por mostrar $(h) \Rightarrow (a)$.
      
      Sean $x, y, z$ en $V_G$ tales que $P$ es una $xy$-trayectoria que no pasa por $z$.
      Como $G$ es conexa, entonces existe una $xy$-trayectoria $R$ y una $yz$-trayectoria
      $S$, tales que, $T = RS$, luego hemos encontrado dos $xy$-trayectorias distintas que
      en partícular forman un ciclo, además este ciclo es único porque esto pasa con cualquier
      vértice y entre ellos, \textit{i.e.}, si seguimos escogiendo vértices de $V_G$ veremos que
      todos están en un ciclo y que están relacionados de esta manera con todos los vértices, de
      lo anterior $G$ es un único ciclo y concluimos que $G$ es un bloque.
    \end{itemize}

    De lo anterior y por silogismo hipotético en los incisos anteriores se concluye que la
    caracterización se cumple.
  \end{proof}
\end{enumerate}

\end{document}
