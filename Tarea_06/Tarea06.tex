\documentclass{article}

% Symbols
%\usepackage{recycle}
\usepackage{amsfonts, amsthm}
\usepackage{upgreek}
\usepackage{physics}
\usepackage{cancel}
\usepackage{amssymb, latexsym, amsmath}

% Proof
\renewcommand*{\proofname}{\textbf{Demostraci\'on:}}

% Graphics
\usepackage{graphicx}
\usepackage{pgf}

% Color a letras.
%\usepackage[usenames,dvipsnames,svgnames,table]{xcolor}

% Tikz
\usepackage{tkz-graph}
\usepackage{tikz}
\usetikzlibrary{arrows,automata}
\usepackage{tikz}
\usetikzlibrary{arrows,automata}
% Def. Dr. César.
\usetikzlibrary{shapes,calc}
\tikzstyle{edge}=[shorten <=2pt, shorten >=2pt, >=stealth, line width=1.1pt]
\tikzstyle{blueE}=[shorten <=2pt, shorten >=2pt, >=stealth, line width=1.5pt, blue]
\tikzstyle{blackV}=[circle, fill=black, minimum size=6pt, inner sep=0pt, outer sep=0pt]
\tikzstyle{blueV}=[circle, fill=blue, draw, minimum size=6pt, line width=0.75pt, inner sep=0pt, outer sep=0pt]
\tikzstyle{redV}=[circle, fill=red, draw, minimum size=6pt, line width=0.75pt, inner sep=0pt, outer sep=0pt]
\tikzstyle{redSV}=[semicircle, fill=red, minimum size=3pt, inner sep=0pt, outer sep=0pt, rotate=225]
\tikzstyle{blueSV}=[semicircle, fill=blue, minimum size=3pt, inner sep=0pt, outer sep=0pt, rotate=225]
\tikzstyle{blackSV}=[semicircle, fill=black, minimum size=3pt, inner sep=0pt, outer sep=0pt, rotate=225]
\tikzstyle{vertex}=[circle, draw, minimum size=6pt, line width=0.75pt, inner sep=0pt, outer sep=0pt]

% Margins
\addtolength{\voffset}{-1cm}
\addtolength{\hoffset}{-1cm}
\addtolength{\textwidth}{2cm}
\addtolength{\textheight}{2cm}

%Header-Footer
\usepackage{fancyhdr}
\renewcommand{\headrulewidth}{1pt}

\newcommand{\set}[1]{%
  \left\{ #1 \right\}%
}

%\pagenumbering{gobble} -- Este comando
%                       -- quita el número de página.
\footskip = 50pt
\renewcommand{\headrulewidth}{1pt}

\pagestyle{fancyplain}

\begin{document}
\title{UNIVERSIDAD AUT\'ONOMA DE M\'EXICO\\ Facultad de Ciencias}
\author{Autores:
  \\ Fernanda Villaf\'an Flores
  \\ Fernando Alvarado Palacios
  \\ Adri\'an Aguilera Moreno}
\date{}
\maketitle
\begin{center}
  \includegraphics[scale=0.20]{../Imagen/Portada.jpg}\\[0.4cm]
  \Large
  \bf{Gr\'aficas y Juegos}
  \normalsize
\end{center}
\newpage
\fancyhead[r]{ Gr\'aficas y Juegos 2022-1}
%%%%%%%%%%%%%%%%%%%%%%%%%%%%%%%%%%%%%%%%%%%%%%%%%%%%%
\section*{\LARGE{Tarea 6}}
\begin{enumerate}
\item Sea $G$ una gr\'afica conexa no euleriana.   Demuestre que
  las siguientes afirmaciones son equivalentes.
  \begin{enumerate}
  \item Hay un paseo euleriano en $G$.

  \item Hay exactamente dos v\'ertices de grado impar en $G$.

  \item Existe una familia de ciclos ajenos por aristas dos
    a dos $\{ C_i \}_{i=1}^k$ y un paseo $P$ tal que
    $E_G = E_P \cup \bigcup_{i=1}^k E_{C_i}$.
  \end{enumerate}

\item Sea $D$ una digr\'afica conexa. Demuestre que $D$ es
  euleriana si y s\'olo si  para cada $v \in V_D$, se tiene
  $d^+(v) = d^-(v)$.

  \begin{proof} 
    $\Longrightarrow$ Sea D una digrafica conexa e eucliriana $\rightarrow$ Por definicion de grafica eucliriana existe un circuito euclidiano que une a todos los vertices, llamemosle C a este circuito $\rightarrow$ sea x perteneciente a $V_D$ el inicio de este circuito $\rightarrow$ C$=(x,v_i, v_{i+1},...,v_{i+n},u)$ con i y n pertenecientes a los naturales, $\rightarrow$ como todos los vertices son consecutivos notemos que cada vertice de C es cola y cabeza para dos flechas distintas en el circuito $\rightarrow$ para todo $V_k$ que pertenece a C existe $d^+$ y $d^-$ que unene a $V_k$ con sus vertices adyacentes  $V_{k-1}$ y $V_{k+1}$  $\rightarrow$ cada vertice de la trayectoria C tendrá  una "arista" $d^+$ y una $d^-$, ya que D es par y por construccion de C.
    Por lo tanto $d^+=d^-$ ya que para todo vertice de D se puen sumar el numero de veces que aparecen en la trayectoria C y preservará la igualdad anterior.
    
    
    
    $\Longleftarrow$ Sea D conexa y para toda v que pertenece a $V_D$ se tiene que $d^+ = d^- \rightarrow$ para toda v que pertenece a $V_D$ existe al menos una $d^+$ y una $d^-$, por lo que para todo v que pertenece a $V_D$ v es mayor igual a 2, pero el grado de V siempre debe ser par, ya que tenemos la igualdad $d^+=d^- \rightarrow$ D es par, por lo tanto por teorema visto en clase tenemos que D es una grafica Eucliriana 
    
    \end{proof}



\item La digr\'afica de {\em de Bruijn-Good} $BG_n$ tiene como
  conjunto de v\'ertices al conjunto de todas las sucesiones
  binarias de longitud $n$, y donde el v\'ertice $a_1 a_2
  \cdots a_n$ es adyacente al v\'ertice $b_1 b_2 \cdots b_n$
  si y s\'olo si $a_{i+1} = b_i$ para $1 \le i \le n-1$.  Demuestre
  que $BG_n$ es una digr\'afica euleriana de orden $2^n$ y
  di\'ametro dirigido $n$.

\item Demuestre que existe una forma de ordenar todas las fichas
  de domin\'o en un ciclo (respetando las reglas del juego).
  ?`C\'omo generalizar\'ia este resultado para domin\'os con
  $n$ puntos? (el domin\'o est\'andar es el de $6$ puntos).

\item Sean $G$ una gr\'afica euleriana no trivial
  y $u \in V_G$. Demuestre que todo paseo en $G$
  que inicia en $u$ se puede extender a un circuito
  euleriano si y s\'olo si $G-u$ es ac\'iclica.

\end{enumerate}

\section*{Puntos Extra}

\begin{enumerate}
\item Una digr\'afica $D$ es {\em balanceada} si $|d^+(v) - d^-(v)|
  \le 1$, para cada $v \in V$.   Demuestre que toda gr\'afica tiene
  una orientaci\'on balanceada.

\item Una sucesi\'on circular $s_1 s_2 \cdots s_{2^n}$ de ceros
  y unos es llamada una {\em sucesi\'on de de Bruijn-Good}
  de orden $n$ si las $2^n$ subsucesiones $s_i s_{i+1} \cdots
  s_{i+n-1}$, $1 \le i \le 2^n$ (con los sub\'indices tomados
  m\'odulo $2^n$ son distintas, y por lo tanto constituyen todas
  las posibles sucesiones binarias de longitud $n$.   Por ejemplo,
  la sucesi\'on $00011101$ es una una sucesi\'on de de Bruijn-Good
  de orden tres.   Muestre como encontrar un de estas sucesiones
  para cualquier orden $n$ utilizando un circuito euleriano dirigido
  en la gr\'afica de de Bruijn-Good $BG_{n-1}$. Justifique su
  respuesta.

\item Sea $G$ una gr\'afica conexa, y sea $X$ el conjunto
  de v\'ertices de $G$ de grado impar.   Suponga que
  $|X| = 2k$, con $k \ge 1$.
  \begin{enumerate}
  \item Demuestre que hay $k$ paseos ajenos por
    aristas $Q_1, \dots, Q_k$ en $G$ tales que
    $E_G = \bigcup_{i=1}^k E_{Q_i}$.

  \item Deduza que $G$ contiene $k$ paseos ajenos
    por aristas que conectan a los v\'ertices de $X$
    en pares.
  \end{enumerate}

\end{enumerate}
\end{document}
