\documentclass{article}

% Symbols
\usepackage{amsfonts, amsthm}
\usepackage{upgreek}
\usepackage{physics}
\usepackage{cancel}
\usepackage{amssymb, latexsym, amsmath}

%Algorithms
\usepackage[ruled,lined,linesnumbered,commentsnumbered]{algorithm2e}

%% Identación
\setlength{\parindent}{0cm}

% Código
\newcommand{\code}[1]{\textcolor{white!25!black}{\texttt{#1}}}
\usepackage{listings}

%AMS
\usepackage{amsthm}
\newtheorem{algo-thm}{Algoritmo}

% Proof
\renewcommand*{\proofname}{\textbf{Demostraci\'on:}}
% Theorem
\newtheorem*{theorem}{Teorema}

% Graphics
\usepackage{graphicx}
\usepackage{pgf}

% Color a letras.
%\usepackage[usenames,dvipsnames,svgnames,table]{xcolor}

% Tikz
\usepackage{tkz-graph}
\usepackage{tikz}
\usetikzlibrary{arrows,automata}
\usepackage{tikz}
\usetikzlibrary{arrows,automata}
%\usetikzlibrary[topaths]

% Def. Dr. César.
\usetikzlibrary{shapes,calc}
\tikzstyle{edge}=[shorten <=2pt, shorten >=2pt, >=stealth, line width=1.1pt]
\tikzstyle{blueE}=[shorten <=2pt, shorten >=2pt, >=stealth, line width=1.5pt, blue]
\tikzstyle{blackV}=[circle, fill=black, minimum size=6pt, inner sep=0pt, outer sep=0pt]
\tikzstyle{blueV}=[circle, fill=blue, draw, minimum size=6pt, line width=0.75pt, inner sep=0pt, outer sep=0pt]
\tikzstyle{redV}=[circle, fill=red, draw, minimum size=6pt, line width=0.75pt, inner sep=0pt, outer sep=0pt]
\tikzstyle{redSV}=[semicircle, fill=red, minimum size=3pt, inner sep=0pt, outer sep=0pt, rotate=225]
\tikzstyle{blueSV}=[semicircle, fill=blue, minimum size=3pt, inner sep=0pt, outer sep=0pt, rotate=225]
\tikzstyle{blackSV}=[semicircle, fill=black, minimum size=3pt, inner sep=0pt, outer sep=0pt, rotate=225]
\tikzstyle{vertex}=[circle, draw, minimum size=6pt, line width=0.75pt, inner sep=0pt, outer sep=0pt]

% Margins
\addtolength{\voffset}{-1.5cm}
\addtolength{\hoffset}{-1.5cm}
\addtolength{\textwidth}{3cm}
\addtolength{\textheight}{3cm}

%Header-Footer
\usepackage{fancyhdr}
\renewcommand{\headrulewidth}{1pt}

\newcommand{\set}[1]{
  \left\{ #1 \right\}
}

%\pagenumbering{gobble} -- Este comando
%                       -- quita el número de página.
\footskip = 50pt
\renewcommand{\headrulewidth}{1pt}

\pagestyle{fancyplain}

\begin{document}
\title{UNIVERSIDAD AUT\'ONOMA DE M\'EXICO\\ Facultad de Ciencias}
\author{Autores:
  \\ Fernanda Villaf\'an Flores
  \\ Fernando Alvarado Palacios
  \\ Adri\'an Aguilera Moreno}
\date{}
\maketitle
\begin{center}
  \includegraphics[scale=0.20]{../Imagen/Portada.jpg}\\[0.4cm]
  \Large
  \bf{Gr\'aficas y Juegos}
  \normalsize
\end{center}
\newpage
\fancyhead[r]{ Gr\'aficas y Juegos 2022-1}
%%%%%%%%%%%%%%%%%%%%%%%%%%%%%%%%%%%%%%%%%%%%%%%%%%%%%
\section*{\LARGE{Tarea 9 y 10}}
\section*{Puntos Extra}
\begin{enumerate}
\item (2 puntos) Sea $G$ una gr\'afica conexa y $e$ una arista de $G$ que
  no sea un lazo. Exhiba una biyecci\'on entre el conjunto de \'arboles
  generadores de $G$ que contienen a $e$ y el conjunto de \'arboles
  generadores de $G / e$.

\item (2 puntos) Sea $T$ un \'arbol de DFS de una gr\'afica conexa no
  trivial $G$, y sea $v$ la raiz de un bloque $B$ de $G$.   Demuestre
  que el grado de $v$ en $T \cap B$ es uno.

\item (2 puntos)  Si $f$ es la funci\'on de tiempo de entrada del
  algoritmo DFS, defina $f^\ast\colon V \to \mathbb{N}$ de la siguiente
  forma.   Si alg\'un ancestro propio de $v$ puede ser alcanzado desde
  $v$ mediante una trayectoria dirigida que consista de flechas del
  \'arbol (posiblemente ninguna) seguida de una flecha que no est\'a en
  el \'arbol (que va hacia arriba), $f^\ast (v)$ se define como el menor
  valor de $f$ de un ancestro de este tipo; si no, $f^\ast (v) = f(v)$.
  Observe que un v\'ertice $v$ es la ra\'iz de un bloque si y s\'olo si
  tiene un hijo $w$ tal que $f^\ast (w) \ge f(v)$.   Modifique el algoritmo
  DFS para que regrese los v\'ertices de corte y los bloques de una
  gr\'afica conexa.

\item (2 puntos) Sea $T$ un \'arbol \'optimo en una gr\'afica conexa
  ponderada $(G,w)$ (con pesos positivos), y sean $x$ y $y$ v\'ertices
  adyacentes en $T$. Demuestre que la trayectoria $xTy = xy$ es una
  $xy$-trayectoria de peso m\'inimo en $G$.

\item (2 puntos) Demuestre que si todos los pesos de una gr\'afica
  ponderada $G$ son distintos, entonces $G$ tiene un \'unico \'arbol
  \'optimo.

\item (2 puntos) Modifique el algoritmo de Bor\r uvka-Kruskal para que en cada
  iteraci\'on v\'ertices en la misma componente del bosque $F$ reciban el
  mismo color y v\'ertices en componentes distintas reciban colores distintos.

\item (2 puntos) Demuestre que el problema de encontrar un \'arbol generador
  de peso m\'aximo en una gr\'afica conexa puede resolverse eligiendo
  iterativamente una arista de peso m\'aximo, con la condici\'on de que la
  subgr\'afica resultante siga siendo un bosque. (Proponga un algoritmo y
  demuestre que es correcto.)

\item (2 puntos) Escriba una versi\'on del algoritmo BFS para digr\'aficas.
  Utilice esta versi\'on de BFS dirigida para describir un algoritmo que
  encuentre un ciclo dirigido de longitud m\'inima en una digr\'afica.    Su
  versi\'on dirigida de BFS debe de correr en tiempo $\mathcal{O} (|V| +
  |E|)$, y el algoritmo para encontrar el ciclo dirigido m\'as corto debe
  correr en tiempo a lo m\'as $\mathcal{O} (|V|^2 + |V| |E|)$.
\end{enumerate}

\end{document}
