\documentclass{article}

% Symbols
\usepackage{amsfonts, amsthm}
\usepackage{upgreek}
\usepackage{physics}
\usepackage{cancel}
\usepackage{amssymb, latexsym, amsmath}

%Algorithms
\usepackage[ruled,lined,linesnumbered,commentsnumbered]{algorithm2e}

%% Identación
\setlength{\parindent}{0cm}

% Código
\newcommand{\code}[1]{\textcolor{white!25!black}{\texttt{#1}}}
\usepackage{listings}

%AMS
\usepackage{amsthm}
\newtheorem{algo-thm}{Algoritmo}

% Proof
\renewcommand*{\proofname}{\textbf{Demostraci\'on:}}
% Theorem
\newtheorem*{theorem}{Teorema}

% Graphics
\usepackage{graphicx}
\usepackage{pgf}

% Color a letras.
%\usepackage[usenames,dvipsnames,svgnames,table]{xcolor}

% Tikz
\usepackage{tkz-graph}
\usepackage{tikz}
\usetikzlibrary{arrows,automata}
\usepackage{tikz}
\usetikzlibrary{arrows,automata}
%\usetikzlibrary[topaths]

% Def. Dr. César.
\usetikzlibrary{shapes,calc}
\tikzstyle{edge}=[shorten <=2pt, shorten >=2pt, >=stealth, line width=1.1pt]
\tikzstyle{blueE}=[shorten <=2pt, shorten >=2pt, >=stealth, line width=1.5pt, blue]
\tikzstyle{blackV}=[circle, fill=black, minimum size=6pt, inner sep=0pt, outer sep=0pt]
\tikzstyle{blueV}=[circle, fill=blue, draw, minimum size=6pt, line width=0.75pt, inner sep=0pt, outer sep=0pt]
\tikzstyle{redV}=[circle, fill=red, draw, minimum size=6pt, line width=0.75pt, inner sep=0pt, outer sep=0pt]
\tikzstyle{redSV}=[semicircle, fill=red, minimum size=3pt, inner sep=0pt, outer sep=0pt, rotate=225]
\tikzstyle{blueSV}=[semicircle, fill=blue, minimum size=3pt, inner sep=0pt, outer sep=0pt, rotate=225]
\tikzstyle{blackSV}=[semicircle, fill=black, minimum size=3pt, inner sep=0pt, outer sep=0pt, rotate=225]
\tikzstyle{vertex}=[circle, draw, minimum size=6pt, line width=0.75pt, inner sep=0pt, outer sep=0pt]

% Margins
\addtolength{\voffset}{-1.5cm}
\addtolength{\hoffset}{-1.5cm}
\addtolength{\textwidth}{3cm}
\addtolength{\textheight}{3cm}

%Header-Footer
\usepackage{fancyhdr}
\renewcommand{\headrulewidth}{1pt}

\newcommand{\set}[1]{
  \left\{ #1 \right\}
}

%\pagenumbering{gobble} -- Este comando
%                       -- quita el número de página.
\footskip = 50pt
\renewcommand{\headrulewidth}{1pt}

\pagestyle{fancyplain}

\begin{document}
\title{UNIVERSIDAD AUT\'ONOMA DE M\'EXICO\\ Facultad de Ciencias}
\author{Autores:
  \\ Fernanda Villaf\'an Flores
  \\ Fernando Alvarado Palacios
  \\ Adri\'an Aguilera Moreno}
\date{}
\maketitle
\begin{center}
  \includegraphics[scale=0.20]{../Imagen/Portada.jpg}\\[0.4cm]
  \Large
  \bf{Gr\'aficas y Juegos}
  \normalsize
\end{center}
\newpage
\fancyhead[r]{ Gr\'aficas y Juegos 2022-1}
%%%%%%%%%%%%%%%%%%%%%%%%%%%%%%%%%%%%%%%%%%%%%%%%%%%%%
\section*{\LARGE{Tarea 11}}
\begin{enumerate}
\item Sea $D$ una digr\'afica con una funci\'on $f\colon A \to \mathbb{R}$.
  Demuestre que
  \begin{enumerate}
  \item $\sum \{ f^+(v) \colon\  v \in V \} = \sum \{ f^-(v) \colon\ v \in V
    \}$,

  \item si $f$ es un $(x,y)$-flujo, el flujo neto $f^+(x) - f^-(x)$ que sale
    de $x$ es igual al flujo neto $f^-(y) - f^+(y)$ que entra a $y$.
  \end{enumerate}

\item \begin{enumerate}
\item Demuestre que, para cualquier flujo $f$ en una red $N$ y cualquier
  conjunto $X \subseteq V$, $$\sum_{v \in X} (f^+(v) - f^-(v)) = f^+(X) -
  f^-(X).$$

\item D\'e un ejemplo de un flujo $f$ en una red tal que $\sum_{v \in X}
  f^+(v) \ne f^+(X)$ y $\sum_{v \in X} f^-(v) \ne f^-(X)$.
\end{enumerate}

\item Sea $N(X,Y)$ con conjunto de fuentes $X = \{ x_i \}_{i = 1}^k$ y
  conjunto de sumideros $Y = \{ y_j \}_{j = 1}^\ell$. Construya una nueva red
  $N'(x,y)$ de la siguiente forma.
  \begin{itemize}
  \item Agregue dos nuevos v\'ertices $x$ y $y$.

  \item Para cada $i \in \{ 1, \dots, k \}$, una a $x$ con $x_i$ con una
    flecha de capacidad $c^+ (x_i)$.

  \item Para cada $j \in \{ 1, \dots, \ell \}$, una a $y_j$ con $y$ con una
    flecha de capacidad $c^- (y_j)$.
  \end{itemize}
  Para cualquier flujo $f$ en $N$, considere la funci\'on $f'$ definida sobre
  $N'$ como:
  \[f'(a) = \left\{
  \begin{array}{ll}
    f(a) & \textnormal{si } a \textnormal{ es una flecha de } N \\
    f^+(v) & \textnormal{si } a = (x,v) \\
    f^-(v) & \textnormal{si } a = (v,y).
  \end{array}
  \right.\]
  \begin{enumerate}
  \item Demuestre que $f'$ es un flujo en $N'$ con el mismo valor que $f$.

  \item Demuestre, rec\'iprocamente, que la restricci\'on de un flujo en
    $N'$ al conjunto de flechas de $N$ es un flujo en $N$ del mismo valor.
  \end{enumerate}

\item Sea $N(x,y)$ una red que no contenga $xy$-trayectorias dirigidas.
  Demuestre, sin utilizar el teorema del Flujo M\'aximo-Corte M\'inimo, que el
  valor de un flujo m\'aximo, y la capacidad de un corte m\'inimo en $N$ son
  ambos cero.

\item Sea $N$ una red con un flujo $f$.   Demuestre que si $P$ es una
  trayectoria $f$-incrementable con $\varepsilon (P) = \varepsilon$, entonces
  la funci\'on dada por \[
  f'(a) = \left\{
  \begin{array}{ll}
    f(a) + \varepsilon & \textnormal{si } a \in A_P \text{ y } \to \\
    f(a) - \varepsilon & \textnormal{si } a \in A_P \text{ y }
    \leftarrow \\
    f(a) & \textnormal{si } a \notin A_P.
  \end{array}
  \right.\]
  es un flujo con valor estrictamente mayor que $f$.

\end{enumerate}

\end{document}
