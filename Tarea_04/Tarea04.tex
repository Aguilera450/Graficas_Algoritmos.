\documentclass{article}

% Symbols
%\usepackage{recycle}
\usepackage{amsfonts, amsthm}
\usepackage{upgreek}
\usepackage{physics}
\usepackage{cancel}
\usepackage{amssymb, latexsym, amsmath}

% Proof
\renewcommand*{\proofname}{\textbf{Demostraci\'on:}}

% Graphics
\usepackage{graphicx}
\usepackage{pgf}

% Color a letras.
%\usepackage[usenames,dvipsnames,svgnames,table]{xcolor}

% Tikz
\usepackage{tkz-graph}
\usepackage{tikz}
\usetikzlibrary{arrows,automata}
\usepackage{tikz}
\usetikzlibrary{arrows,automata}
% Def. Dr. César.
\usetikzlibrary{shapes,calc}
\tikzstyle{edge}=[shorten <=2pt, shorten >=2pt,
  >=stealth, line width=1.1pt]
\tikzstyle{blueE}=[shorten <=2pt, shorten >=2pt,
  >=stealth, line width=1.5pt, blue]
\tikzstyle{blackV}=[circle, fill=black,
  minimum size=6pt,
  inner sep=0pt, outer sep=0pt]
\tikzstyle{blueV}=[circle, fill=blue, draw,
  minimum size=6pt, line width=0.75pt,
  inner sep=0pt, outer sep=0pt]
\tikzstyle{redV}=[circle, fill=red, draw,
  minimum size=6pt, line width=0.75pt,
  inner sep=0pt, outer sep=0pt]
\tikzstyle{redSV}=[semicircle, fill=red, minimum
  size=3pt, inner sep=0pt, outer sep=0pt,
  rotate=225]
\tikzstyle{blueSV}=[semicircle, fill=blue, minimum
  size=3pt, inner sep=0pt, outer sep=0pt,
  rotate=225]
\tikzstyle{blackSV}=[semicircle, fill=black, minimum
  size=3pt, inner sep=0pt, outer sep=0pt,
  rotate=225]
\tikzstyle{vertex}=[circle, draw, minimum size=6pt,
  line width=0.75pt, inner sep=0pt,
  outer sep=0pt]

% Nueva Def.
\newcommand{\kaspressknodel}[2]{\pgfmathtruncatemacro{\mymod}{#1+1}
  \foreach \X in {0,...,#1}
           {\node[bullet,label={right:$\boldsymbol{(2,\X)}$}] (v2\X) at 
             (4,{1.2*(\X-(#1+0)/2)}){};
             \node[bullet,label={left:$\boldsymbol{(1,\X)}$}] (v1\X) at 
             (1.5,{1.2*(\X-(#1+0)/2)}){};}
           \foreach \X [evaluate=\X as \NextX using {int(mod(\X+3,\mymod))},
             evaluate=\X as \AnotherX using {int(mod(\X+1,\mymod))},] in {0,...,#1}
                    {\draw[thick] (v1\X) -- (v2\NextX);
                      \draw[dashed] (v1\X) -- (v2\AnotherX);
                      \draw (v1\X) -- (v2\X);} 
}

% Margins
\addtolength{\voffset}{-1.5cm}
\addtolength{\hoffset}{-1.5cm}
\addtolength{\textwidth}{3cm}
\addtolength{\textheight}{3cm}

%Header-Footer
\usepackage{fancyhdr}
\renewcommand{\headrulewidth}{1pt}

\newcommand{\set}[1]{%
  \left\{ #1 \right\}%
}

%\pagenumbering{gobble} -- Este comando
%                       -- quita el número de página.
\footskip = 50pt
\renewcommand{\headrulewidth}{1pt}

\pagestyle{fancyplain}

\begin{document}
\title{UNIVERSIDAD AUT\'ONOMA DE M\'EXICO\\ Facultad de Ciencias}
\author{Autores:
  \\ Fernanda Villaf\'an Flores
  \\ Fernando Alvarado Palacios
  \\ Adri\'an Aguilera Moreno}
\date{}
\maketitle
\begin{center}
  \includegraphics[scale=0.20]{../Imagen/Portada.jpg}\\[0.4cm]
  \Large
  \bf{Gr\'aficas y Juegos}
  \normalsize
\end{center}
\newpage
\fancyhead[r]{ Gr\'aficas y Juegos 2022-1}
%%%%%%%%%%%%%%%%%%%%%%%%%%%%%%%%%%%%%%%%%%%%%%%%%%%%%
\section*{\LARGE{Tarea 4}}

\begin{enumerate}
  %%%%%%%%%%%%%%%%%%%%%%%%%%%%%%%%%%%% Ejercicio 1. %%%%%%%%%%%%%%%%%%%%%%%%%%%%%
  \item Sea $G$ una gr\'afica no trivial.   Demuestre
    que $G$ es una trayectoria si y s\'olo si $G$ es
    un \'arbol con exactamente dos v\'ertices de
    grado $1$.
    


    \begin{proof} (Reducción al absurdo)
      (------->) Pd G es una trayectoria $\Longrightarrow$ G es un arbol con exactamente 2 vértices de grado 1
      
      Pd G es una trayectoria  y G no es un árbol 
      
      Pd G es una trayectoria  y G no es conexa o contiene un ciclo 
      
      Caso 1 G no es conexa
      
      Si G no es conexa $\Longrightarrow$ Existen  u,v que pertenecen a G tal que u,v no son adyacente $\Longrightarrow$ G no puede ser una trayectoria (Lo que es una contradicción ya que G es una trayectoria)
      
      Caso 2 G contiene un ciclo
      
      Si G contiene un ciclo $\Longrightarrow$  G no es un orden lineal  $\Longrightarrow$ G no es trayectoria (Lo que es una contradicción ya que G es una trayectoria)
      
      Por lo tanto G es una trayectoria y G es un Árbol $\Longrightarrow$ (---tengo dudas aqui---) Existe G con tansolo 2 vertices de grado 2, pero tambien existen G con mas vertices que lo cumplen 
      
      
      (<-------) Sea G un arbol con exactamente 2 vertices de grado 1  $\Longrightarrow$ G es un orden lineal por lo tanto G es una trayectoria
      
      \end{proof}


    %%%%%%%%%%%%%%%%%%%%%%%%%%%%%%%%%%%% Ejercicio 2. %%%%%%%%%%%%%%%%%%%%%%%%%%%
  \item \begin{enumerate}
    %~~~~~~~~~~~~~~~~~~~~~~~~~~~~~~~~~~~~~ 2 (a).
    \item Demuestre que cada \'arbol con grado m\'aximo
      $\Delta > 1$ tiene al menos $\Delta$ hojas.
      \renewcommand\qedsymbol{QED}
      \begin{proof}
      Sea $G$ un \'arbol y sea $x \in V_G$ tal que $d(x) = \Delta$ (notar que $x$
      no necesariamente es el \'unico que cumple tener grado igual a $\Delta$, por
      lo que se tomar\'a alguno que cumpla esto), entonces $x$ tiene exactamente
      $\Delta$ vecinos, por la caracterizaci\'on de \'arbol, sabemos que $G$ es
      ac\'iclico y por tanto los caminos que parten desde $x$ (tienen a $x$
      como v\'ertice inicial) hacia algunos de sus $\Delta$ vecinos, no tienen
      v\'ertices en com\'un que sean distintos de $x$ (caso contrario, habr\'ia $2$
      trayectorias que tienen inicio en $x$ a las que llamaremos $w_1$ y $w_2$, y
      adem\'as tienen en com\'un al menos un v\'ertice $v$ y por tanto $w_1 w_2$ es
      un ciclo!! que est\'a contenido en $G$), luego, como $x$ tiene $\Delta$ vecinos,
      entonces podemos tomar al menos $\Delta$ trayectorias distintas entre ellas
      tales que terminen en alg\'un v\'ertice $u_i$ ($1 \leq i \leq \Delta$) y por
      tanto cada $u_i$ es una hoja de $G$, como hemos encontrado $\Delta$ hojas
      podemos concluir que el \'arbol $G$ con grado m\'aximo $\Delta$ tiene al menos
      $\Delta$ hojas.
      \end{proof}
      %~~~~~~~~~~~~~~~~~~~~~~~~~~~~~~~~~~~ 2 (c).
    \item Construya, para cada elecci\'on de $n$ y $\Delta$,
      con $2\le \Delta < n$, un \'arbol de orden $n$ con
      exactamente $\Delta$ hojas.
      
      \textbf{\textit{Soluci\'on:}} Sea $G$ un \'arbol, usemos el resultado anterior,
      y garantizamos que, como $G$ tiene alg\'un $x \in V_G : d(x) = \Delta$, entonces
      $G$ tiene al menos $\Delta$ hojas. Luego, part\'icularmente en los \'arboles de
      exactamente $\Delta$ hojas, cada camino $W_i$ que tenga como v\'ertice inicial a
      $x$ no tendr\'a bifurcaciones, esto es, $x W_i v_i$ es la \'unica manera de llegar
      de $x$ a $v_i$ y adem\'as $W_i$ es una trayectoria (y por definici\'on de trayectoria,
      no tendr\'a bifurcaciones), como $x$ es v\'ertice inicial de exactamente $\Delta$
      trayectorias $W_i$, entonces hay exactamente $\Delta$ hojas en $G$. En resumen, $G$
      tendr\'a exactamente $\Delta$ v\'ertices de grado $1$, a continuaci\'on se muestra
      un ejemplo que trata de ser lo m\'as general posible:
      
      %%%%%%%%%%%%%%%%%%%%%%%%%%%%%%%%%%%%%%%%%%%%%%%%%%%%%%%%
      \begin{figure}[ht!]
        \centering
        \begin{tikzpicture}
          %%%%%  Componente izquierda  %%%%%
          %\foreach \i in {0,1}
          %\node(\i) [vertex] at (\i*1, -2){};
          \node(0) [blueV, label=270:$v_1$] at (0, -2){};
          \node(1) [vertex] at (1, -2){};
          \draw[edge] (0) to (1);
          \node (L) at (-0.5,-1.5){$G$};
          
          %% Suspensivos.
          \begin{scope}[xshift=1cm]
            \foreach \i in {1,...,3}
            \fill (\i*0.7, -2) circle (1pt);
          \end{scope}
          
          %% x y vecino.
          \begin{scope}[xshift=4cm]
            \node(3) [vertex] at (0, -2){};
            \node(4) [redV, label=270:$x$] at (1, -2){};
            \draw[edge] (3) to (4);
          \end{scope}
          
          %%  Nodos, esto se puede mejorar con un foreach.
          \begin{scope}[xshift=5cm]
            \node(5) [vertex] at (3, 1){};
            \node(6) [vertex] at (3, 0){};
            \foreach \i in {1,...,3}
            \fill (3, -\i*1) circle (1pt);
            \node(7) [vertex] at (3, -4){};
            \node(8) [vertex] at (3, -5){};  
          \end{scope}

          %% Aristas de x hacia la derecha.
          \draw[edge] (4) to (5);
          \draw[edge] (4) to (6);
          \draw[edge] (4) to (7);
          \draw[edge] (4) to (8);
          
          %% Parte derecha.
          \begin{scope}[xshift=8cm]
            \foreach \j in {1,0,...,-5}
            \foreach \i in {1,...,3}
            \fill (\i*0.7, \j) circle (1pt);
            \node(9)  [vertex] at (3, 1){};
            \node(10) [blueV, label=0:$v_2$] at (4, 1){};
            \draw[edge] (9) to (10);
            \node(11) [vertex] at (3, 0){};
            \node(12) [blueV, label=0:$v_3$] at (4, 0){};
            \draw[edge] (11) to (12);
            \foreach \j in {3,4}
            \foreach \i in {-1,-2,-3}
            \fill (\j, \i) circle (1pt);
            \node(13)  [vertex] at (3, -4){};
            \node(14) [blueV, label=0:$v_{\Delta -1}$] at (4, -4){};
            \draw[edge] (13) to (14);
            \node(15) [vertex] at (3, -5){};
            \node(16) [blueV, label=0:$v_{\Delta}$] at (4, -5){};
            \draw[edge] (15) to (16);
          \end{scope}
        \end{tikzpicture}
      \end{figure}
      %%%%%%%%%%%%%%%%%%%%%%%%%%%%%%%%%%%%%%%%%%%%%%%%%%%%%%%%
      
      donde los $v_i$'s son las hojas, para $1 \leq i \leq \Delta$.
      \hfill $\square$
  \end{enumerate}
    
    %%%%%%%%%%%%%%%%%%%%%%%%%%%%%%%%%%%% Ejercicio 3. %%%%%%%%%%%%%%%%%%%%%%%%%%%%
  \item Un {\em centro} en una gr\'afica es un v\'ertice
    $u$ tal que $\max_{v \in V} d(u, v)$ es m\'inima.
    Demuestre que un \'arbol tiene exactamente un centro
    o dos centros adyacentes.
    
    %%%%%%%%%%%%%%%%%%%%%%%%%%%%%%%%%%%% Ejercicio 4. %%%%%%%%%%%%%%%%%%%%%%%%%%%%
  \item Demuestre o brinde un contraejemplo: Toda
    gr\'afica con menos aristas que v\'ertices tiene
    una componente que es un \'arbol.
    
    %%%%%%%%%%%%%%%%%%%%%%%%%%%%%%%%%%%% Ejercicio 5. %%%%%%%%%%%%%%%%%%%%%%%%%%%%
  \item Un {\em hidrocarburo saturado} es una
    mol\'ecula $C_mH_n$ en la que cada \'atomo de
    carbono tiene cuatro enlaces, cada
    \'atomo de hidr\'ogeno tiene un enlace, y
    ninguna sucesi\'on de enlaces forma un ciclo.
    Demuestre que para cualquier entero positivo
    $m$, la mol\'ecula $C_mH_n$ existe s\'olo si
    $n = 2m + 2$.
    


    \begin{proof} Demostración por inducción sobre m

      Paso base (m=1)
      
      Por definición de hidrocarburo saturado $C_1 k_4 \Longrightarrow$  4 = 2(1)+2 por lo tanto para m=1 se cumple que n=2m+2
      
      Hipótesis de inducción m = k, si  $C_k H_n \Longrightarrow$ supongamos n=2k+2
      
      Paso Inductivo
      
      Pd m=k+1
      
      Por hipótesis de inducción tenemos que $C_k H_n \Longrightarrow$ n=2k+2 y por paso base $c_1 K_4 \Longrightarrow$ 4=2(1)+2 $\Longrightarrow$ sean r que pertenece a los Naturales sin el 0, sea $C_r$ y $C_1$ donde  $C_r$ pertenece a $C_k H_n$  y $C_1$ pertenece a $C_1 k_4$ tal que r pertenece a ${1,2,3,...,k}$  $\Longrightarrow$  eliminemos 1 hidrógeno a $C_r$ y $C_1 \Longrightarrow C_k H_{n-1}$ y $C_1 k_3$ son iguales a n-1=2k+1 ...(1) y 3=2(1)+1 ...(2) $\Longrightarrow$ uniendo $C_k H_{n-1}$ y $C_1 k_3$ mediante los vertices $C_r$ y $C_1$ $\Longrightarrow C_{k+1} H_{m}$ seria igual a la suma de (1) y (2) $\Longrightarrow$ n+2 = 2(k) + 2(1) +2 $\Longrightarrow$  n+2 = 2(k+1)+2
      $\Longrightarrow$ m=n+1 $\Longrightarrow$  n+2 = 2(k+1)+2. Por lo tanto para $C_{k+1} H_{m}$ m=2(k+1)+2
      
      Por lo tanto,  para todo m que pertenece a Naturales sin el cero $C_m H_n$ tal que n=2m+2 
      
      
      \end{proof}



    %%%%%%%%%%%%%%%%%%%%%%%%%%%%%%%%%%%% Ejercicio 6. %%%%%%%%%%%%%%%%%%%%%%%%%%%%
  \item Demuestre que una sucesi\'on $(d_1, \dots,
    d_n)$ de enteros positivos es la sucesi\'on de
    grados de un \'arbol si y s\'olo si
    $\sum_{i=1}^n d_i = 2(n-1)$.
    \renewcommand\qedsymbol{QED}
    \begin{proof}
      Para este ejercicio analicemos $2$ posibles casos:
      \begin{itemize}
      \item[$\Rightarrow$)] Dada la sucesi\'on de grados $(d_1, \dots, d_n)$ de un
        \'arbol, entonces $\sum_{i=1}^n d_i = 2(n-1)$.
        
        Sabemos que para cualquier gr\'afica pasa que
        \[
        \sum_{i=1}^n d_i = 2|E|
        \]
        Como en un \'arbol se cumple que $|E| = |V| - 1$\footnote{Prop. $2.2.5$ en las
          notas de clase}, entonces $|E| = n - 1$ y por tanto
        \[
        \sum_{i=1}^n d_i = 2(n - 1)
        \]
        
      \item[$\Leftarrow$)] Dada una gr\'afica $G$ donde se cumple que
        $\sum_{i=1}^n d_i = 2(n-1)$, entonces $(d_1, \dots, d_n)$ es la
        sucesi\'on de grados en un arbol.
        
        Veamos que $G$ no tiene v\'ertices aislados, pues en caso de tenerlos
        supongamos sin p\'erdida de generalidad que $x \in V_G$ es un
        v\'ertice aislado, entonces $G - \{x\}$ no contiene v\'ertices aislados
        y $|E_{G - \{x\}}| = 2|V_{G - \{x\}}|$ !!, lo que implica que $G$ contiene como
        subgr\'afica inducida a alg\'un ciclo (pues la cantidad de v\'ertices
        ser\'ia de al menos la cantidad de v\'ertices) y se deja de cumplir que
        $\sum_{i=1}^n d_i = 2(n-1)$ !!.
        
        De la misma manera que la anterior podemos observar que todos los v\'ertices
        de $G$ no pueden tener al menos grado $2$, pues suponer esto nos genera
        al menos un ciclo y esto implicar\'ia que $\sum_{i=1}^n d_i = 2(n)$!!, y
        claramente $2n \not= 2(n - 1)$.
        
        Luego, hay exactamente $2$ v\'ertices de grado $1$ en $G$, para esto
        llamemos $x, y \in V_G$ a los v\'ertices de grado $1$, entonces en
        $G -\{x, y\}$ se cumple que $\sum_{i=1}^n d_i = 2(n-2)$ y al anexarle
        exactamente $\{x, y\}$ tendremos que
        \begin{eqnarray*}
          \sum_{i=1}^n d_i &=& 2(n-2) + 2\\
          &=& 4n - 4 + 2\\
          &=& 4n - 2\\
          &=& 2(n - 1)
        \end{eqnarray*}
        en caso contrario, se deja de cumplir lo anterior y es por esto que
        se puede garantizar que estos son \'unicos.
        
        Como $G$ es ac\'iclica (por el argumento dado anteriormente), podemos
        deducir que $G$ es conexa, pues de no serlo habr\'ia m\'as de $2$
        v\'ertices con grado $1$ y ya observamos que este no es un caso posible.
        Luego, $G$ es un \'arbol y por el ejercicio $1$ de esta tarea tenemos
        que $G$ es, part\'icularmente, una trayectoria.
      \end{itemize}
      
      De los casos analizados concluimos que una sucesi\'on $(d_1, \dots,
      d_n)$ de enteros positivos es la sucesi\'on de grados de un \'arbol
      si y s\'olo si $\displaystyle \sum_{i=1}^n d_i = 2(n-1)$.
    \end{proof}
\end{enumerate}
  %%%%%%%%%%%%%%%%%%%%%%%%%%%%%%%%%%%% Extras %%%%%%%%%%%%%%%%%%%%%%%%%%%%%%%%
\section*{Puntos Extra}
\begin{enumerate}
  %-------------------------------------- Extra 1.
  \item Para una gr\'afica conexa $G$ definimos la
    gr\'afica de \'arboles de $G$, $\mathcal{T}_G$,
    como la gr\'afica que tiene por v\'ertices a
    todos los \'arboles generadores de $G$, y tal que,
    si $S, T \in V_{\mathcal{T}_G}$, entonces $ST$ es
    una arista de $\mathcal{T}_G$ si y s\'olo si
    existen aristas $e \in E_S - E_T$ y $f \in E_T -
    E_S$ tales que $(S - e) + f = T$.   Demuestre que
    $\mathcal{T}_G$ es conexa.
    %-------------------------------------- Extra 2.
  \item Sea $T$ un \'arbol arbitrario con $k+1$ v\'ertices.
    Demuestre que si $G$ es simple y $\delta \ge k$,
    entonces $G$ tiene una subgr\'afica isomorfa a $T$.
    %-------------------------------------- Extra 3.
  \item Sea $\mathcal{T}$ una familia de sub\'arboles
    de un \'arbol $T$.   Deduzca, por inducci\'on sobre
    $|\mathcal{T}|$, que si cualesquiera dos elementos
    de $\mathcal{T}$ tienen un v\'ertice en com\'un,
    entonces hay un v\'ertice de $T$ que est\'a en
    todos los elementos de $\mathcal{T}$.
      %-------------------------------------- Extra 4.
  \item \begin{enumerate}
    \item Determine todos los arboles $T$ tales que
      $\overline T$ tambi\'en es un \'arbol.
      
      \textbf{\textit{Soluci\'on:}} 
      \begin{itemize}
      \item[$\cdot$)]  Si $|V_T| = 1$, entonces por vacuidad se cumple
        el enunciado y terminamos.
      \item[$\cdot \cdot$)] Si $|V_T| > 1$, entonces sabemos existen $|V_T| -1$
        aristas para $T$ y que a lo más ${|V_T| \choose 2}$ si $T$ fuera
        completa. Notar que $E_{\overline{T}} = {|V_T| \choose 2} - (|V_T| -1)$,
        como queremos que $\overline{T}$ se un árbol, entonces se debe cumplir
        la siguiente igualdad
        \begin{eqnarray*}
          |V_T| -1 &=& {|V_T| \choose 2} - (|V_T| -1)\\
          \Leftrightarrow 2 \cdot (|V_T| -1) &=& {|V_T| \choose 2}\\
          \Leftrightarrow 2 \cdot (|V_T| -1) &=& \frac{|V_T| \cdot (|V_T| -1)}{2}\\
          \Leftrightarrow 4 \cdot \cancel{(|V_T| -1)} &=& |V_T| \cdot \cancel{(|V_T| -1)}\\
          \Rightarrow |V_T| &=& 4
        \end{eqnarray*}
        
        y del ejercicio $1$ de la tarea $1$ sabemos que hay $11$ gráficas
        de orden $4$ no isomorfas entre sí y sólo $2$ de esas son árboles,
        una es $P_4$ y la otra es el árbol tal que uno de sus vértices es
        de grado $3$, pero en este último su complemento no es un árbol.
        Luego  $T = P_4$ y este es la único  salvo isomorfismo.
        \hfill $\square$
        
      \end{itemize}
      
    \item Determine todas las gr\'aficas de orden al
      menos cuatro tales que la subgr\'afica inducida por
      cualesquiera tres de sus v\'ertices es un \'arbol.
      
      \textbf{\textit{Soluci\'on:}} 
      Veamos que si $|V| > 4$, se cumple cualquiera de las siguientes condiciones:
      \begin{itemize}
      \item[$\cdot$)] Si la gráfica es incompleta existen al menos dos vértices
        que no se conectan mediante una arista, si no existe trayectoria entre estos,
        entonces existe una "elección" de $3$ vértices tales que son inducidos de
        la gráfica original y no son un árbol.
        
      \item[$\cdot \cdot$)] Si la gráfica es completa, entonces hay $3$ vértices que
        al inducirlos en la gráfica original forman un $3$-ciclo y por tanto no son un árbol.
      \end{itemize}
      De lo anterior la única gráfica que cumple con el enunciado es un $4$-ciclo
      que no contenga como gráfica inducida un $3$-ciclo. \hfill $\square$
    \end{enumerate}
    
\end{enumerate}

\end{document}
